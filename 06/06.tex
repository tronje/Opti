\newcommand{\authorinfotitle}{Carolin Konietzny, 6523939, Gruppe 3\\ Tronje Krabbe, 6435002, Gruppe 7 \\ Julian Tobergte, 6414935, Gruppe 5}
\newcommand{\authorinfo}{Carolin Konietzny, Julian Tobergte, Tronje Krabbe}
\newcommand{\titleinfo}{Optimierung 06}
\newcommand{\qed}{\square}
\newcommand{\tilD}{($\widetilde{\mathrm{D}}$)}

\documentclass [a4paper,11pt]{article}
\usepackage[german,ngerman]{babel}
\usepackage[utf8]{inputenc}
\usepackage[T1]{fontenc}
\usepackage{lmodern}
\usepackage{amssymb}
\usepackage{mathtools}
\usepackage{amsmath}
\usepackage{enumerate}
%\usepackage{breqn}
\usepackage{fancyhdr}
\usepackage{multicol}
\usepackage{eurosym}

\usepackage[a4paper,left=2cm,width=13cm,right=3cm]{geometry}

% for dem plots
\usepackage{pgfplots}
%\pgfplotsset{compat=1.10}
\usepgfplotslibrary{fillbetween}
% ---

\author{\authorinfotitle}
\title{\titleinfo}
\date{\today}

\pagestyle{fancy}
\fancyhf{}
\fancyhead[R]{\authorinfo}
\fancyhead[L]{Opti Hausaufgaben}
\fancyfoot[C]{\thepage}

\begin{document}
\maketitle
    \begin{enumerate}
        % Aufgabe 1
        \item[\textbf{1.}]
        \begin{enumerate}
        \item[\textbf{a)}]
            (P):
            \begin{align*}
            \begin{alignedat}{6}
            & \text{maximiere } & 9x_1 &\ - &\ 5x_2 &\ - &\ 4x_3 & & \\
            & \rlap{unter den Nebenbedingungen} & & & & & & & \\
            &&  3x_1 &\ + &\ 3x_2 &\ - &\ x_3 &\ \leq &\ 4\ \\
            &&  5x_1 &\ + &\ 3x_2 &\ + &\ x_3 &\ \leq &\ 6\ \\
            &&  -x_1 &\ + &\ 3x_2 &\ + &\ x_3 &\ \leq &\ 2\ \\
            &&  3x_1 &\ - &\ 4x_2 &\ - &\ x_3 &\ \leq &\ 2\ \\
            && & & & & \llap{$x_1,x_2,x_3$} &\ \geq &\ 0.
            \end{alignedat}
            \end{align*}

            \begin{enumerate}
            \item[(i)]
                (D):
                \begin{align*}
                \begin{alignedat}{6}
                & \text{minimiere } & 4y_1 &\ + &\ 6y_2 &\ + &\ 2y_3 &\ + &\ 2y_4 & & \\
                & \rlap{unter den Nebenbedingungen} & & & & & & & & & \\
                &&  3y_1 &\ + &\ 5y_2 &\ - &\  y_3 &\ + &\ 3y_4 &\ \geq &\  9\ \\
                &&  3y_1 &\ + &\ 3y_2 &\ + &\ 3y_3 &\ - &\ 4y_4 &\ \geq &\ -5\ \\
                &&  -y_1 &\ + &\  y_2 &\ + &\  y_3 &\ - &\  y_4 &\ \geq &\ -5\ \\
                && & & & & \llap{$y_1,y_2,y_3,y_4$} &\ \geq &\ 0.
                \end{alignedat}
                \end{align*}

            \item[(ii)]
                \begin{align*}
                (x_1^*,x_2^*,x_3^*) &= (\frac{30}{29},\frac{8}{29},0) \\
                % & z   &\ = &\ \frac{230}{29} &\ - &\ \frac{85}{29}x_3 &\ - &\ \frac{52}{29}x_7 &\ - &\ \frac{21}{29}x_5
                (y_1^*,y_2^*,y_3^*,y_4^*) &= (0,\frac{21}{29},0,\frac{52}{29})
                \end{align*}
            \item[(iii)]
                \begin{align*}
                \begin{alignedat}{6}
                %& \text{maximiere } & 4 \cdot 0 &\ + &\ 6 \cdot \frac{21}{29} &\ + &\ 2 \cdot 0 &\ + &\ 2 \cdot \frac{52}{29} & & \\
                %& \rlap{unter den Nebenbedingungen} & & & & & & & & & \\
                &&  3 \cdot 0 &\ + &\ 5 \cdot \frac{21}{29} &\ - &\         0 &\ + &\ 3 \cdot \frac{52}{29} &\    = &\  9\ \\
                &&  3 \cdot 0 &\ + &\ 3 \cdot \frac{21}{29} &\ + &\ 3 \cdot 0 &\ - &\ 4 \cdot \frac{52}{29} &\    = &\ -5\ \\
                &&          0 &\ + &\         \frac{21}{29} &\ + &\         0 &\ - &\         \frac{52}{29} &\ \geq &\ -5\ \\
                && & & & & \llap{$0,\frac{21}{29},0,\frac{52}{29}$} &\ \geq &\ 0.
                \end{alignedat}
                \end{align*}
                Kommt hin. Man sollte sich merken, dass die ersten beiden Nebenbedingungen mit Gleichheit erfüllt sind.
            \item[(iv)]
                $ 4 \cdot 0 + 6 \cdot \frac{21}{29} + 2 \cdot 0 + 2 \cdot \frac{52}{29} = \frac{230}{29} $\\
                Die optimale Lösung von (D) war ebenfalls $\frac{230}{29}$. Somit ist nach dem Dualitätssatz gezeigt, dass die ermittelte Lösung des dualen Problems
                auch eine optimale Lösung desselben ist.
            \item[(v)]
                Die ersten beiden dualen Ungleichungen sind unter der gegebenen Lösung mit Gleichheit erfüllt (siehe (iii)). Die entsprechenden Variablen in der
                optimalen Lösung von (D) sind nicht gleich Null. Soweit sind die komplementären Schlupfbedingungen erfüllt. Die letzte duale Ungleichung ist nicht
                durch Gleichheit erfüllt, aber die entsprechende Variable ist auch gleich Null. Somit sind die Bedingungen des komplementären Schlupfes für (D)
                und (P) erfüllt.
            \end{enumerate}
        \item[\textbf{b)}]
            (P):
            \begin{align*}
            \begin{alignedat}{6}
            & \text{maximiere } & 4x_1 &\ - &\ 13x_2 &\ - &\ 9x_3 &\ + &\  x_4 & & \\
            & \rlap{unter den Nebenbedingungen} & & & & & & & & & \\
            &&  x_1 &\ + &\ 3x_2 &\ + &\  x_3 &\ + &\ x_4 &\ \leq &\ 4\ \\
            &&  x_1 &\ - &\ 7x_2 &\ - &\ 3x_3 &\ - &\ x_4 &\ \leq &\ 1\ \\
            && & & & & & & \llap{$x_1,x_2,x_3,x_4$} &\ \geq &\ 0.
            \end{alignedat}
            \end{align*}

            \begin{enumerate}
                \item[(i)]
                    (D):
                    \begin{align*}
                    \begin{alignedat}{4}
                    & \text{minimiere } & 4y_1 &\ + &\ y_2 & & \\
                    & \rlap{unter den Nebenbedingungen} & & & & & \\
                    &&  y_1 &\ + &\  y_2 &\ \geq &\ 4\ \\
                    && 3y_1 &\ - &\ 7y_2 &\ \geq &\ -13\ \\
                    &&  y_1 &\ - &\ 3y_2 &\ \geq &\ -9\ \\
                    &&  y_1 &\ - &\  y_2 &\ \geq &\ 1\ \\
                    && & & \llap{$y_1,y_2$} &\ \geq &\ 0.
                    \end{alignedat}
                    \end{align*}

                \item[(ii)]
                    % & z   &\ = &\ \frac{23}{2} &\ - &\ 7x_3 &\ - &\ \frac{3}{2}x_6 &\ - &\  \frac{5}{2}x_5 &\ - &\ 10x_2
                    $ (y_1^*,y_2^*) = (\frac{5}{2},\frac{3}{2}) $

                \item[(iii)]
                    \begin{align*}
                    \begin{alignedat}{4}
                    %& \text{maximiere } & 4y_1 &\ + &\ y_2 & & \\
                    %& \rlap{unter den Nebenbedingungen} & & & & & \\
                    &&  \frac{5}{2} &\ + &\  \frac{3}{2} &\ = &\ 4\ \\
                    && 3 \cdot \frac{5}{2} &\ - &\ 7 \cdot \frac{3}{2} &\ \geq &\ -13\ \\
                    &&  \frac{5}{2} &\ - &\ 3 \cdot \frac{3}{2} &\ \geq &\ -9\ \\
                    &&  \frac{5}{2} &\ - &\  \frac{3}{2} &\ = &\ 1\ \\
                    && & & \llap{$\frac{5}{2},\frac{3}{2}$} &\ \geq &\ 0.
                    \end{alignedat}
                    \end{align*}
                    Die erhaltene Lösung von (D) ist also auch zulässig. Man beachte, dass die erste und vierte Ungleichung tatsächlich mit Gleichheit erfüllt
                    sind.
                \item[(iv)]
                    Die optimale Lösung von (P) war $\frac{23}{2}$.\\
                    $ 4 \cdot \frac{5}{2} + \frac{3}{2} = \frac{23}{2} $\\
                    Die Lösung ist also nach dem Dualitätssatz optimal.

                \item[(v)]
                    Die erste und vierte duale Ungleichung ist jeweils mit Gleichheit erfüllt. Die entsprechenden Variablen sind aber auch ungleich Null.
                    Die zweite und dritte duale Ungleichung ist jeweils mit Ungleichheit erfüllt. Aber die entsprechenden Variablen sind gleich Null.
                    Die komplementären Schlupfbedingungen sind also erfüllt.


            \end{enumerate}
        \end{enumerate}
        % Aufgabe 2
        \item[\textbf{2.}]
            (D):
            \begin{align*}
            \begin{alignedat}{12}
            & \text{minimiere } & -y_1 &\ + &\ y_2 &\ + &\ 6y_3 &\ + &\ 6y_4 &\ - &\ 3y_5 &\ + &\ 6y_6 & & \\
            & \rlap{unter den Nebenbedingungen} & & & & & & & & & & & & & \\
            && -3y_1 &\ + &\ y_2 &\ - &\ 2y_3 &\ + &\ 9y_4 &\ - &\ 5y_5 &\ + &\ 7y_6 &\ \geq &\ -1\ \\
            &&   y_1 &\ - &\ y_2 &\ + &\ 7y_3 &\ - &\ 4y_4 &\ + &\ 2y_5 &\ - &\ 3y_6 &\ \geq &\ -2\ \\
            && & & & & & & & & & & \llap{$y_1,y_2,y_3,y_4,y_5,y_6$} &\ \geq &\ 0.
            \end{alignedat}
            \end{align*}\\

            \tilD:
            \begin{align*}
            \begin{alignedat}{12}
            & \text{maximiere } & y_1 &\ - &\ y_2 &\ - &\ 6y_3 &\ - &\ 6y_4 &\ + &\ 3y_5 &\ - &\ 6y_6 & & \\
            & \rlap{unter den Nebenbedingungen} & & & & & & & & & & & & & \\
            && 3y_1 &\ - &\ y_2 &\ + &\ 2y_3 &\ - &\ 9y_4 &\ + &\ 5y_5 &\ - &\ 7y_6 &\ \leq &\ 1\ \\
            && -y_1 &\ + &\ y_2 &\ - &\ 7y_3 &\ + &\ 4y_4 &\ - &\ 2y_5 &\ + &\ 3y_6 &\ \leq &\ 2\ \\
            && & & & & & & & & & & \llap{$y_1,y_2,y_3,y_4,y_5,y_6$} &\ \geq &\ 0.
            \end{alignedat}
            \end{align*}
            \newpage
            Hieraus folgern wir unser\\
            \underline{Starttableau}:
            \begin{align*}
            \begin{alignedat}{12}
            && y_7 &\ = &\ 1 &\ - &\ 3y_1 &\ + &\ y_2 &\ - &\ 2y_3 &\ + &\ 9y_4 &\ - &\ 5y_5 &\ + &\ 7y_6 \\
            && y_8 &\ = &\ 2 &\ + &\  y_1 &\ - &\ y_2 &\ + &\ 7y_3 &\ - &\ 4y_4 &\ + &\ 2y_5 &\ - &\ 3y_6 \\
            &\rlap{\rule{7.8cm}{.1pt}} &&&&&&&&&&&&&&&\\
            && z   &\ = &    &    &\  y_1 &\ - &\ y_2 &\ - &\ 6y_3 &\ - &\ 6y_4 &\ + &\ 3y_5 &\ - &\ 6y_6 
            \end{alignedat}
            \end{align*}

            1. Iteration mit Eingangsvariable $y_5$ und Ausgangsvariable $y_7$:
            \begin{align*}
            \begin{alignedat}{12}
            && y_5 &\ = &\  \frac{1}{5} &\ - &\ \frac{3}{5}y_1 &\ + &\ \frac{1}{5}y_2 &\ - &\  \frac{2}{5}y_3 &\ + &\ \frac{9}{5}y_4 &\ + &\ \frac{7}{5}y_6 &\ - &\ \frac{1}{5}y_7 \\
            && y_8 &\ = &\ \frac{12}{5} &\ + &\ \frac{1}{5}y_1 &\ - &\ \frac{3}{5}y_2 &\ + &\ \frac{31}{5}y_3 &\ - &\ \frac{2}{5}y_4 &\ - &\ \frac{1}{5}y_6 &\ - &\ \frac{2}{5}y_7 \\
            &\rlap{\rule{9cm}{.1pt}} &&&&&&&&&&&&&&&\\
            && z   &\ = &\  \frac{3}{5} &\ - &\ \frac{4}{5}y_1 &\ - &\ \frac{2}{5}y_2 &\ - &\ \frac{36}{5}y_3 &\ - &\ \frac{3}{5}y_4 &\ - &\ \frac{9}{5}y_6 &\ - &\ \frac{3}{5}y_7 
            \end{alignedat}
            \end{align*}

            Hier terminiert der Simplexalgorithmus. Die optimale Lösung von \tilD \ ist $\frac{3}{5}$, während die optimale Lösung von (P) $-\frac{3}{5}$ ist.
            Diese Invertierung stammt daher, dass wir das duale Problem von (P) zuerst in Standardform gebracht hatten.
            Die Variablen nehmen in den optimalen Lösungen die folgenden Werte an:
            \begin{align*}
                (y_1,y_2,y_3,y_4,y_5,y_6) &= (0,0,0,0,\frac{1}{5},0)\\
                (x_1,x_2) &= (\frac{3}{5},0)
            \end{align*}


    \end{enumerate}
\end{document}