\newcommand{\authorinfotitle}{Carolin Konietzny, 6523939, Gruppe 3\\ Tronje Krabbe, 6435002, Gruppe 7 \\ Julian Tobergte, 6414935, Gruppe 5}
\newcommand{\authorinfo}{Carolin Konietzny, Tronje Krabbe, Julian Tobergte}
\newcommand{\titleinfo}{Optimierung 02 27.10.2014}
\newcommand{\qed}{\square}

\documentclass [a4paper,11pt]{article}
\usepackage[german,ngerman]{babel}
\usepackage[utf8]{inputenc}
\usepackage[T1]{fontenc}
\usepackage{lmodern}
\usepackage{amssymb}
\usepackage{mathtools}
\usepackage{amsmath}
\usepackage{enumerate}
%\usepackage{breqn}
\usepackage{fancyhdr}
\usepackage{multicol}

\usepackage[a4paper,left=2cm,width=13cm,right=3cm]{geometry}

% for dem plots
\usepackage{pgfplots}
%\pgfplotsset{compat=1.10}
\usepgfplotslibrary{fillbetween}
% ---

\author{\authorinfotitle}
\title{\titleinfo}
\date{\today}

\pagestyle{fancy}
\fancyhf{}
\fancyhead[R]{\authorinfo}
\fancyhead[L]{Opti Hausaufgaben}
\fancyfoot[C]{\thepage}

\begin{document}
\maketitle
    \begin{enumerate}
        % Aufgabe 1
        \item[\textbf{1.}]
            Starttableau:
            \begin{align*}
            \begin{alignedat}{6}
            & x_4 &\ = &\ 7 &\ - &\ x_1 &\ - &\ 3x_2 &\ - &\ 2x_3 \\
            & x_5 &\ = &\ 4 &\ - &\ x_1 &\ - &\ 2x_2 &\ - &\  x_3 \\
            & x_6 &\ = &\ 5 &    &      &\ - &\ 3x_2 &\ - &\ 2x_3 \\
            &\rlap{\rule{5cm}{.1pt}} &&&&&&&&\\
            & z   &\ = &    &    &\ 2x_1 &\ + &\  4x_2 &\ + &\ 3x_3 
            \end{alignedat}
            \end{align*}
            1. Iteration:\\
            Eingangsvariable: $x_2$, da es den größten Koeffizienten in $z$ hat\\
            Ausgangsvariable: $x_6$, da:
            \begin{align*}
                &x_1 = x_3 = 0\\
                & 0 \leq x_4 = 7 - 3x_2 \Rightarrow x_2 \geq \frac{7}{3}\\
                & 0 \leq x_5 = 4 - 2x_2 \Rightarrow x_2 \leq 2\\
                & 0 \leq x_6 = 5 - 3x_2 \Rightarrow x_2 \leq \frac{5}{3} \Rightarrow \text{strengste Beschränkung}
            \end{align*}
            Es folgt:
            \begin{align*}
                & x_2  = \frac{5}{3} - \frac{2}{3} x_3 -\frac{1}{3} x_6\\
                & x_4  = 7 - x_1 - 3\left( \frac{5}{3} - \frac{2}{3} x_3 -\frac{1}{3} x_6 \right) - 2x_3\\
                &    \ = 2 - x_1 + \frac{8}{3}x_3 + \frac{1}{3} x_6\\
                & x_5  = 4 - x_1 - 2\left( \frac{5}{3} - \frac{2}{3} x_3 -\frac{1}{3} x_6 \right) - x_3\\
                &    \ = \frac{2}{3} - x_1 + \frac{1}{3} x_3 + \frac{2}{3} x_6\\
                &   z  = 2x_1 + 4\left( \frac{5}{3} - \frac{2}{3} x_3 -\frac{1}{3} x_6 \right) + 3x_3\\
                &    \ = \frac{20}{3} + 2x_1 + \frac{1}{3} x_3 - \frac{4}{3} x_6
            \end{align*}
            Ergebnis der 1. Iteration:
            \begin{align*}
            \begin{alignedat}{6}
            & x_2 &\ = &\ \frac{5}{3}  &    &       &\ - &\ \frac{2}{3}x_3 &\ - &\ \frac{1}{3}x_6 \\
            & x_4 &\ = &\ 2            &\ - &\  x_1 &\ + &\ \frac{8}{3}x_3 &\ + &\ \frac{1}{3}x_6 \\
            & x_5 &\ = &\ \frac{2}{3}  &\ - &\  x_1 &\ + &\ \frac{1}{3}x_3 &\ + &\ \frac{2}{3}x_6 \\
            &\rlap{\rule{5cm}{.1pt}} &&&&&&&&\\
            & z   &\ = &\ \frac{20}{3} &\ + &\ 2x_1 &\ + &\ \frac{1}{3}x_3 &\ - &\ \frac{4}{3}x_6
            \end{alignedat}
            \end{align*}

        % Aufgabe 2
        \item[\textbf{2.}]


    \end{enumerate}


\end{document}
