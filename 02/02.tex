\newcommand{\authorinfotitle}{Carolin Konietzny, 6523939, Gruppe 3\\ Tronje Krabbe, 6435002, Gruppe 7 \\ Julian Tobergte, 6414935, Gruppe 5}
\newcommand{\authorinfo}{Carolin Konietzny, Tronje Krabbe, Julian Tobergte}
\newcommand{\titleinfo}{Optimierung 02 27.10.2014}
\newcommand{\qed}{\square}

\documentclass [a4paper,11pt]{article}
\usepackage[german,ngerman]{babel}
\usepackage[utf8]{inputenc}
\usepackage[T1]{fontenc}
\usepackage{lmodern}
\usepackage{amssymb}
\usepackage{mathtools}
\usepackage{amsmath}
\usepackage{enumerate}
%\usepackage{breqn}
\usepackage{fancyhdr}
\usepackage{multicol}

\usepackage[a4paper,left=2cm,width=13cm,right=3cm]{geometry}

% for dem plots
\usepackage{pgfplots}
%\pgfplotsset{compat=1.10}
\usepgfplotslibrary{fillbetween}
% ---

\author{\authorinfotitle}
\title{\titleinfo}
\date{\today}

\pagestyle{fancy}
\fancyhf{}
\fancyhead[R]{\authorinfo}
\fancyhead[L]{Opti Hausaufgaben}
\fancyfoot[C]{\thepage}

\begin{document}
\maketitle
    \begin{enumerate}
        % Aufgabe 1
        \item[\textbf{1.}]
            \begin{enumerate}
            \item[a)]
                \underline{Starttableau}:
                \begin{align*}
                \begin{alignedat}{6}
                & x_4 &\ = &\ 7 &\ - &\ x_1 &\ - &\ 3x_2 &\ - &\ 2x_3 \\
                & x_5 &\ = &\ 4 &\ - &\ x_1 &\ - &\ 2x_2 &\ - &\  x_3 \\
                & x_6 &\ = &\ 5 &    &      &\ - &\ 3x_2 &\ - &\ 2x_3 \\
                &\rlap{\rule{5cm}{.1pt}} &&&&&&&&\\
                & z   &\ = &    &    &\ 2x_1 &\ + &\  4x_2 &\ + &\ 3x_3 
                \end{alignedat}
                \end{align*}
                1. Iteration:\\
                Eingangsvariable: $x_2$, da es den größten Koeffizienten in $z$ hat\\
                Ausgangsvariable: $x_6$, da:
                \begin{align*}
                    &x_1 = x_3 = 0\\
                    & 0 \leq x_4 = 7 - 3x_2 \Rightarrow x_2 \geq \frac{7}{3}\\
                    & 0 \leq x_5 = 4 - 2x_2 \Rightarrow x_2 \leq 2\\
                    & 0 \leq x_6 = 5 - 3x_2 \Rightarrow x_2 \leq \frac{5}{3} \Rightarrow \text{stärkste Beschränkung}
                \end{align*}
                Es folgt:
                \begin{align*}
                    & x_2  = \frac{5}{3} - \frac{2}{3} x_3 -\frac{1}{3} x_6\\
                    & x_4  = 7 - x_1 - 3\left( \frac{5}{3} - \frac{2}{3} x_3 -\frac{1}{3} x_6 \right) - 2x_3\\
                    &    \ = 2 - x_1 + \frac{8}{3}x_3 + \frac{1}{3} x_6\\
                    & x_5  = 4 - x_1 - 2\left( \frac{5}{3} - \frac{2}{3} x_3 -\frac{1}{3} x_6 \right) - x_3\\
                    &    \ = \frac{2}{3} - x_1 + \frac{1}{3} x_3 + \frac{2}{3} x_6\\
                    &   z  = 2x_1 + 4\left( \frac{5}{3} - \frac{2}{3} x_3 -\frac{1}{3} x_6 \right) + 3x_3\\
                    &    \ = \frac{20}{3} + 2x_1 + \frac{1}{3} x_3 - \frac{4}{3} x_6
                \end{align*}
                \underline{Ergebnis der 1. Iteration}:
                \begin{align*}
                \begin{alignedat}{6}
                & x_2 &\ = &\ \frac{5}{3}  &    &       &\ - &\ \frac{2}{3}x_3 &\ - &\ \frac{1}{3}x_6 \\
                & x_4 &\ = &\ 2            &\ - &\  x_1 &\ + &\ \frac{8}{3}x_3 &\ + &\ \frac{1}{3}x_6 \\
                & x_5 &\ = &\ \frac{2}{3}  &\ - &\  x_1 &\ + &\ \frac{1}{3}x_3 &\ + &\ \frac{2}{3}x_6 \\
                &\rlap{\rule{5cm}{.1pt}} &&&&&&&&\\
                & z   &\ = &\ \frac{20}{3} &\ + &\ 2x_1 &\ + &\ \frac{1}{3}x_3 &\ - &\ \frac{4}{3}x_6
                \end{alignedat}
                \end{align*}

                2. Iteration:\\
                Eingangsvariable: $x_1$\\
                Ausgangsvariable: $x_5$, da:
                \begin{align*}
                    &x_3 = x_6 = 0\\
                    & 0 \leq 2 - x_1 \Rightarrow x_1 \leq 2 \\
                    & 0 \leq \frac{2}{3} - x_1 \Rightarrow x_1 \leq \frac{2}{3} \Rightarrow \text{stärkste Beschränkung}
                \end{align*}
                Es folgt:
                \begin{align*}
                & x_1 = \frac{2}{3} + \frac{1}{3}x_3 + \frac{2}{3}x_6 - x_5\\
                & x_2 = \frac{5}{3} - \frac{2}{3}x_3 - \frac{1}{3}x_6\\
                & x_4 = 2 - \left( \frac{2}{3} + \frac{1}{3}x_3 + \frac{2}{3}x_6 - x_5 \right) + \frac{8}{3}x_3 + \frac{1}{3}x_6\\
                &   \ = \frac{4}{3} +\frac{7}{3}x_3 - \frac{1}{3}x_6 - x_5\\
                &   z = \frac{20}{3} + 2 \left( \frac{2}{3} + \frac{1}{3}x_3 + \frac{2}{3}x_6 - x_5 \right) + \frac{1}{3}x_3 - \frac{4}{3}x_6\\
                &   \ = 8 + x_3 - 2x_5
                \end{align*}
                \underline{Ergebnis der 2. Iteration}:
                \begin{align*}
                \begin{alignedat}{6}
                & x_1 &\ = &\ \frac{2}{3}  &\ + &\ \frac{1}{3}x_3 &\ + &\ \frac{2}{3}x_6 &\ - &\ x_5 \\
                & x_2 &\ = &\ \frac{5}{3}  &\ - &\ \frac{2}{3}x_3 &\ - &\ \frac{1}{3}x_6 &    & \\
                & x_4 &\ = &\ \frac{4}{3}  &\ + &\ \frac{7}{3}x_3 &\ - &\ \frac{1}{3}x_6 &\ - &\ x_5 \\
                &\rlap{\rule{5cm}{.1pt}} &&&&&&&&\\
                & z   &\ = &\ 8 &\ + &\ x_3 & & &\ - &\ 2x_5
                \end{alignedat}
                \end{align*}

                3. Iteration:\\
                Eingangsvariable: $x_3$\\
                Ausgangsvariable: $x_2$\\
                Es folgt:
                \begin{align*}
                & x_3 = \frac{5}{2} - \frac{1}{2}x_6 - \frac{3}{2}x_2\\
                & x_1 = \frac{2}{3} + \frac{1}{3} \left( \frac{5}{2} - \frac{1}{2}x_6 - \frac{3}{2}x_2 \right) + \frac{2}{3}x_6 - x_5\\
                &   \ = \frac{3}{2} + \frac{1}{6}x_6 - x_5 - \frac{1}{2}x_2\\
                & x_4 = \frac{4}{3} + \frac{7}{3} \left( \frac{5}{2} - \frac{1}{2}x_6 - \frac{3}{2}x_2 \right) - \frac{1}{3}x_6 - x_5\\
                &   \ = \frac{43}{6} -\frac{3}{2}x_6 - x_5 - \frac{7}{2}x_2\\
                &   z =           8 + \left( \frac{5}{2} - \frac{1}{2}x_6 - \frac{3}{2}x_2 \right) - 2x_5\\
                &   \ = \frac{21}{2}- 4x_6 - 2x_5 - 12x_2
                \end{align*}
                \underline{Ergebnis der 3. Iteration}:
                \begin{align*}
                \begin{alignedat}{6}
                & x_3 &\ = &\ \frac{5}{2}  &\ - &\ \frac{1}{2}x_6 &    &                 &\ - &\ \frac{3}{2}x_2 \\
                & x_1 &\ = &\ \frac{3}{2}  &\ + &\ \frac{1}{6}x_6 &\ - &\ x_5            &\ - &\ \frac{1}{2}x_2 \\
                & x_4 &\ = &\ \frac{43}{6} &\ - &\ \frac{3}{2}x_6 &\ - &\ x_5            &\ - &\ \frac{7}{2}x_2 \\
                &\rlap{\rule{5cm}{.1pt}} &&&&&&&&\\
                & z   &\ = &\ \frac{21}{2} &\ - &\ 4x_6 &\ - &\ 2x_5 &\ - &\ 12x_2
                \end{alignedat}
                \end{align*}
                Dieses Tableau liefert die optimale Lösung mit $x_1 = \frac{3}{2}$, $x_2 = 0$, $x_3 = \frac{5}{2}$ und $z=\frac{21}{2}$. \\
            \item[b)]
                \underline{Starttableau}:
                \begin{align*}
                \begin{alignedat}{6}
                & x_4 &\ = &\ 4  &\ - &\ 3x_1 &\ - &\ 3x_2 &\ + &\ x_3 \\
                & x_5 &\ = &\ 6  &\ - &\ 5x_1 &\ - &\ 3x_2 &\ - &\ x_3 \\
                & x_6 &\ = &\ 2  &\ + &\  x_1 &\ - &\ 3x_2 &\ - &\ x_3 \\
                & x_7 &\ = &\ 2  &\ - &\ 3x_1 &\ + &\ 4x_2 &\ + &\ x_3 \\
                &\rlap{\rule{5cm}{.1pt}} &&&&&&&&\\
                & z   &\ = &     &    &\ 9x_1 &\ - &\ 5x_2 &\ - &\ 4x_3
                \end{alignedat}
                \end{align*}

                1. Iteration:\\
                Eingangsvariable: $x_1$\\
                Ausgangsvariable: $x_7$\\
                Es folgt:
                \begin{align*}
                & x_1 = \frac{2}{3} + \frac{4}{3}x_2 + \frac{1}{3}x_3 - \frac{1}{3}x_7\\
                & x_4 = 4 - 3\left( \frac{2}{3} + \frac{4}{3}x_2 + \frac{1}{3}x_3 - \frac{1}{3}x_7 \right) - 3x_2 + x_3\\
                &   \ = 2 - 7x_2 + x_7\\
                & x_5 = 6 - 5\left( \frac{2}{3} + \frac{4}{3}x_2 + \frac{1}{3}x_3 - \frac{1}{3}x_7 \right) - 3x_2 - x_3\\
                &   \ = \frac{8}{3} - \frac{29}{3}x_2 - \frac{8}{3}x_3 + \frac{5}{3}x_7\\
                & x_6 = 2 + \left( \frac{2}{3} + \frac{4}{3}x_2 + \frac{1}{3}x_3 - \frac{1}{3}x_7 \right) - 3x_2 - x_3\\
                &   \ = \frac{8}{3} - \frac{5}{3}x_2 - \frac{2}{3}x_3 - \frac{1}{3}x_7\\
                &   z = 9 \left( \frac{2}{3} + \frac{4}{3}x_2 + \frac{1}{3}x_3 - \frac{1}{3}x_7 \right) - 5x_2 - 4x_3\\
                &   \ = 6 + 7x_2 - x_3 - 3x_7
                \end{align*}
                \underline{Ergebnis der 1. Iteration}:
                \begin{align*}
                \begin{alignedat}{6}
                & x_1 &\ = &\ \frac{2}{3}  &\ + &\  \frac{4}{3}x_2 &\ + &\ \frac{1}{3}x_3 &\ - &\ \frac{1}{3}x_7 \\
                & x_4 &\ = &\ 2            &\ - &\            7x_2 &    &                 &\ + &\ x_7 \\
                & x_5 &\ = &\ \frac{8}{3}  &\ - &\ \frac{29}{3}x_2 &\ - &\ \frac{8}{3}x_3 &\ + &\ \frac{5}{3}x_7 \\
                & x_6 &\ = &\ \frac{8}{3}  &\ - &\  \frac{5}{3}x_2 &\ - &\ \frac{2}{3}x_3 &\ - &\ \frac{1}{3}x_7 \\
                &\rlap{\rule{5cm}{.1pt}} &&&&&&&&\\
                & z   &\ = &\ 6            &\ + &\            7x_2 &\ - &\            x_3 &\ - &\ 3x_7
                \end{alignedat}
                \end{align*}

                2. Iteration:\\
                Eingangsvariable: $x_2$\\
                Ausgangsvariable: $x_5$\\
                Es folgt:
                \begin{align*}
                & x_2 = \frac{8}{29} - \frac{8}{29}x_3 + \frac{5}{29}x_7 - \frac{3}{29}x_5\\
                & x_1 = \frac{2}{3} + \frac{4}{3} \left( \frac{8}{29} - \frac{8}{29}x_3 + \frac{5}{29}x_7 - \frac{3}{29}x_5 \right) + \frac{1}{3}x_3 - \frac{1}{3}x_7\\
                &   \ = \frac{30}{29} - \frac{1}{29}x_3 - \frac{3}{29}x_7 - \frac{4}{29}x_5\\
                & x_4 = 2 - 7 \left( \frac{8}{29} - \frac{8}{29}x_3 + \frac{5}{29}x_7 - \frac{3}{29}x_5 \right)  + x_7 \\
                &   \ = \frac{2}{29} + \frac{56}{29}x_3 - \frac{6}{29}x_7 + \frac{21}{29}x_5\\
                & x_6 = \frac{8}{3} - \frac{5}{3} \left( \frac{8}{29} - \frac{8}{29}x_3 + \frac{5}{29}x_7 - \frac{3}{29}x_5 \right) - \frac{2}{3}x_3 - \frac{1}{3}x_7 \\
                &   \ = \frac{64}{29} - \frac{6}{29}x_3 - \frac{18}{29}x_7 + \frac{5}{29}x_5\\
                &   z = 6 + 7 \left( \frac{8}{29} - \frac{8}{29}x_3 + \frac{5}{29}x_7 - \frac{3}{29}x_5 \right) - x_3 - 3x_7\\
                &   \ = \frac{230}{29} - \frac{85}{29}x_3 - \frac{52}{29}x_7 - \frac{21}{29}x_5
                \end{align*}
                \underline{Ergebnis der 2. iteration}:
                \begin{align*}
                \begin{alignedat}{6}
                & x_2 &\ = &\ \frac{8}{29}   &\ - &\  \frac{8}{29}x_3 &\ + &\  \frac{5}{29}x_7 &\ - &\ \frac{3}{29}x_5 \\
                & x_1 &\ = &\ \frac{30}{29}  &\ - &\  \frac{1}{29}x_3 &\ - &\  \frac{3}{29}x_7 &\ - &\ \frac{4}{29}x_5 \\
                & x_4 &\ = &\ \frac{2}{29}   &\ + &\ \frac{56}{29}x_3 &\ - &\  \frac{6}{29}x_7 &\ + &\ \frac{21}{29}x_5 \\
                & x_6 &\ = &\ \frac{64}{29}  &\ - &\  \frac{6}{29}x_3 &\ - &\ \frac{18}{29}x_7 &\ + &\ \frac{5}{29}x_5 \\
                &\rlap{\rule{6cm}{.1pt}} &&&&&&&&\\
                & z   &\ = &\ \frac{230}{29} &\ - &\ \frac{85}{29}x_3 &\ - &\ \frac{52}{29}x_7 &\ - &\ \frac{21}{29}x_5
                \end{alignedat}
                \end{align*}
                Dieses Tableau liefert die optimale Lösung mit $x_1 = \frac{30}{29}$, $x_2 = \frac{8}{29}$, $x_3 = 0$ und $z = \frac{230}{29}$.
            \end{enumerate}
        % Aufgabe 2
        \item[\textbf{2.}]
            \underline{Starttableau}:
            \begin{align*}
            \begin{alignedat}{6}
            & x_5 &\ = &\ 4 &\ - &\  x_1 &\ - &\  3x_2 &\ - &\  x_3 &\ - &\ x_4 \\
            & x_6 &\ = &\ 1 &\ - &\  x_1 &\ + &\  7x_2 &\ + &\ 3x_3 &\ + &\ x_4 \\
            &\rlap{\rule{6cm}{.1pt}} &&&&&&&&&&\\
            & z   &\ = &    &    &\ 4x_1 &\ - &\ 13x_2 &\ - &\ 9x_3 &\ + &\ x_4
            \end{alignedat}
            \end{align*}
            1. Iteration:\\
            Eingangsvariable: $x_1$\\
            Ausgangsvariable: $x_6$\\
            Es folgt:
            \begin{align*}
            & x_1 = 1 + 7x_2 + 3x_3 + x_4 - x_6\\
            & x_5 = 4 - \left( 1 + 7x_2 + 3x_3 + x_4 - x_6 \right) - 3x_2 - x_3 - x_4\\
            &   \ = 3 - 10x_2 - 4x_3 - 2x_4 + x_6\\
            &   z = 4 \left( 1 + 7x_2 + 3x_3 + x_4 - x_6 \right) - 13x_2 - 9x_3 + x_4\\
            &   \ = 4 + 15x_2 + 3x_3 + 5x_4 - 4x_6
            \end{align*}
            \underline{Ergebnis der 1. Iteration}:
            \begin{align*}
            \begin{alignedat}{6}
            & x_1 &\ = &\ 1 &\ + &\  7x_2 &\ + &\ 3x_3 &\ + &\  x_4 &\ - &\ x_6 \\
            & x_5 &\ = &\ 3 &\ - &\ 10x_2 &\ - &\ 4x_3 &\ - &\ 2x_4 &\ + &\ x_6 \\
            &\rlap{\rule{6cm}{.1pt}} &&&&&&&&&&\\
            & z   &\ = &\ 4 &\ + &\ 15x_2 &\ + &\ 3x_3 &\ + &\ 5x_4 &\ - &\ 4x_6
            \end{alignedat}
            \end{align*}
            2. Iteration:\\
            Eingangsvariable: $x_2$\\
            Ausgangsvariable: $x_5$\\
            Es folgt:
            \begin{align*}
            & x_2 = \frac{3}{10} - \frac{2}{5}x_3 - \frac{1}{5}x_4 + \frac{1}{10}x_6 - \frac{1}{10}x_5\\
            & x_1 = 1 + 7 \left( \frac{3}{10} - \frac{2}{5}x_3 - \frac{1}{5}x_4 + \frac{1}{10}x_6 - \frac{1}{10}x_5 \right) + 3x_3 + x_4 - x_6\\
            &   \ = \frac{31}{10} + \frac{1}{5}x_3 - \frac{2}{5}x_4 - \frac{3}{10}x_6 - \frac{7}{10}x_5\\
            &   z = 4 + 15 \left( \frac{3}{10} - \frac{2}{5}x_3 - \frac{1}{5}x_4 + \frac{1}{10}x_6 - \frac{1}{10}x_5 \right) + 3x_3 + 5x_4 - 4x_6\\
            &   \ = \frac{17}{2} - 3x_3 + 2x_4 - \frac{5}{2}x_6 - \frac{3}{2}x_5
            \end{align*}
            \underline{Ergebnis der 2. Iteration}:
            \begin{align*}
            \begin{alignedat}{6}
            & x_2 &\ = &\  \frac{3}{10} &\ - &\ \frac{2}{5}x_3 &\ - &\ \frac{1}{5}x_4 &\ + &\ \frac{1}{10}x_6 &\ - &\ \frac{1}{10}x_5 \\
            & x_1 &\ = &\ \frac{31}{10} &\ + &\ \frac{1}{5}x_3 &\ - &\ \frac{2}{5}x_4 &\ - &\ \frac{3}{10}x_6 &\ - &\ \frac{7}{10}x_5 \\
            &\rlap{\rule{7cm}{.1pt}} &&&&&&&&&&\\
            & z   &\ = &\ \frac{17}{2} &\ - &\           3x_3 &\ + &\           2x_4 &\ - &\  \frac{5}{2}x_6 &\ - &\ \frac{3}{2}x_5
            \end{alignedat}
            \end{align*}
            3. Iteration:\\
            Eingangsvariable: $x_4$\\
            Ausgangsvariable: $x_2$\\
            Es folgt:
            \begin{align*}
            & x_4 = \frac{3}{2} - 2x_3 + \frac{1}{2}x_6 - \frac{1}{2}x_5 - 5x_2\\
            & x_1 = \frac{31}{10} + \frac{1}{5}x_3 - \frac{2}{5} \left( \frac{3}{2} - 2x_3 + \frac{1}{2}x_6 - \frac{1}{2}x_5 - 5x_2 \right) - \frac{3}{10}x_6 - \frac{7}{10}x_5\\
            &   \ = \frac{5}{2} + x_3 - \frac{1}{2}x_6 - \frac{1}{2}x_5 - 2x_2\\
            &   z = \frac{17}{2} - 3x_3 + 2\left( \frac{3}{2} - 2x_3 + \frac{1}{2}x_6 - \frac{1}{2}x_5 - 5x_2 \right) - \frac{5}{2}x_6 - \frac{3}{2}x_5\\
            &   \ = \frac{23}{2} - 7x_3 - \frac{3}{2}x_6 - \frac{5}{2}x_5 - 10x_2
            \end{align*}
            \underline{Ergebnis der 3. Iteration}:
            \begin{align*}
            \begin{alignedat}{6}
            & x_4 &\ = &\  \frac{3}{2} &\ - &\ 2x_3 &\ + &\ \frac{1}{2}x_6 &\ - &\ \frac{1}{2}x_5 &\ - &\ 5x_2 \\
            & x_1 &\ = &\  \frac{5}{2} &\ + &\  x_3 &\ - &\ \frac{1}{2}x_6 &\ - &\ \frac{1}{2}x_5 &\ - &\ 2x_2 \\
            &\rlap{\rule{7cm}{.1pt}} &&&&&&&&&&\\
            & z   &\ = &\ \frac{23}{2} &\ - &\ 7x_3 &\ - &\ \frac{3}{2}x_6 &\ - &\  \frac{5}{2}x_5 &\ - &\ 10x_2
            \end{alignedat}
            \end{align*}
            Dieses Tableau ist die optimale Lösung mit $x_1 = \frac{5}{2}$, $x_2 = 0$, $x_3 = 0$, $x_4 = \frac{3}{2}$ und $z = \frac{23}{2}$.
    \end{enumerate}


\end{document}
