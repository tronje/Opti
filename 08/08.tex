\newcommand{\authorinfotitle}{Carolin Konietzny, 6523939, Gruppe 3\\ Tronje Krabbe, 6435002, Gruppe 7 \\ Julian Tobergte, 6414935, Gruppe 5}
\newcommand{\authorinfo}{Carolin Konietzny, Julian Tobergte, Tronje Krabbe}
\newcommand{\titleinfo}{Optimierung 08}
\newcommand{\qed}{\square}
\newcommand{\tilD}{($\widetilde{\mathrm{D}}$)}

\documentclass [a4paper,11pt]{article}
\usepackage[german,ngerman]{babel}
\usepackage[utf8]{inputenc}
\usepackage[T1]{fontenc}
\usepackage{lmodern}
\usepackage{amssymb}
\usepackage{mathtools}
\usepackage{amsmath}
\usepackage{enumerate}
%\usepackage{breqn}
\usepackage{fancyhdr}
\usepackage{multicol}
\usepackage{eurosym}

\usepackage[a4paper,left=2cm,width=13cm,right=3cm]{geometry}

% for dem plots
\usepackage{pgfplots}
%\pgfplotsset{compat=1.10}
\usepgfplotslibrary{fillbetween}
% ---

\author{\authorinfotitle}
\title{\titleinfo}
\date{\today}

\pagestyle{fancy}
\fancyhf{}
\fancyhead[R]{\authorinfo}
\fancyhead[L]{Opti Hausaufgaben}
\fancyfoot[C]{\thepage}

\begin{document}
\maketitle
    \begin{enumerate}
        % Aufgabe 1
        \item[\textbf{1.}]
        \begin{enumerate}
        \item[a)]
            (D):
            \begin{align*}
            \begin{alignedat}{6}
            & \text{minimiere } & -y_1 &\ + &\ 2y_2 &\ - &\ 3y_3 & & \\
            & \rlap{unter den Nebenbedingungen} & & & & & & & \\
            && -y_1 &\ + &\ 6y_2 &\ - &\  y_3 &\   =  &\  1\ \\
            && 3y_1 &\ + &\  y_2 &\ - &\  y_3 &\   =  &\  3\ \\
            &&  y_1 &\ + &\ 4y_2 &\ + &\ 2y_3 &\ \geq &\ -3\ \\
            && -y_1 &\ - &\ 2y_2 &\ + &\  y_3 &\ \geq &\  2\ \\
            && 7y_1 &\ + &\  y_2 &\ - &\ 2y_3 &\ \geq &\  1\ \\
            && & & & & \llap{$y_1,y_3$} &\ \geq &\ 0
            \end{alignedat}
            \end{align*}
        \item[b)]
            (D):
            \begin{align*}
            \begin{alignedat}{10}
            & \text{maximiere } & -3y_1 &\ + &\ 9y_2 &\ + &\ 5y_3 &\ + &\ 8y_4 &\ + &\ 4y_5 &\ - &\ y_6 &\ - &\ 10y_7 &\ + &\ 9y_8 & & \\
            & \rlap{unter den Nebenbedingungen} & & & & & & & & & & & & & & & & & \\
            &&  -y_1 &\ + &\  y_2 &\ + &\  y_3 &\ + &\ 2y_4 &\ + &\  y_5 &\ + &\ 4y_6 &\ - &\  4y_7 &\ + &\  y_8 &\ \geq &\  5 \\
            &&  5y_1 &\ + &\ 4y_2 &    &       &\ + &\ 4y_4 &\ - &\ 3y_5 &\ - &\ 3y_6 &\ + &\  3y_7 &\ + &\ 2y_8 &\ \geq &\ -1 \\
            &&  -y_1 &\ + &\  y_2 &\ + &\  y_3 &\ - &\  y_4 &\ + &\  y_5 &    &       &\ - &\  5y_7 &\ + &\  y_8 &\   =  &\ 1 \\
            && -2y_1 &    &       &    &       &\ + &\  y_4 &    &       &    &       &\ + &\   y_7 &\ + &\ 7y_8 &\   =  &\ 2 \\
            && & & & & & & & & & & & & & & \llap{$y_1,y_3,y_6,y_7,y_8$} &\ \geq &\ 0
            \end{alignedat}
            \end{align*}

        \end{enumerate}

        % Aufgabe 2
        \item[\textbf{2.}]
            \begin{enumerate}
            \item[a)]
            \begin{enumerate}
                \item[(i)]
                    \underline{Starttableau}:
                    \begin{align*}
                    \begin{alignedat}{6}
                    && x_3 &\ = &\  100 &\ - &\   x_1 &\ - &\   x_2 \\
                    && x_4 &\ = &\ 4000 &\ - &\ 10x_1 &\ - &\ 50x_2 \\
                    &\rlap{\rule{4.8cm}{.1pt}} &&&&&&&\\
                    && z   &\ = &       &    &\ 40x_1 &\ + &\ 70x_2
                    \end{alignedat}
                    \end{align*}

                    Erste Iteration:\\
                    Eingangsvariable: $x_2$\\
                    Ausgangsvariable: $x_4$
                    \begin{align*}
                    \begin{alignedat}{6}
                    && x_2 &\ = &\   80 &\ - &\ \frac{1}{5}x_1 &\ - &\ \frac{1}{50}x_4 \\
                    && x_3 &\ = &\   20 &\ - &\ \frac{4}{5}x_1 &\ + &\ \frac{1}{50}x_4 \\
                    &\rlap{\rule{4.8cm}{.1pt}} &&&&&&&\\
                    && z   &\ = &\ 5600 &\ + &\ 26x_1 &\ - &\ \frac{7}{5}x_4
                    \end{alignedat}
                    \end{align*}

                    Zweite Iteration:\\
                    Eingangsvariable: $x_1$\\
                    Ausgangsvariable: $x_3$
                    \begin{align*}
                    \begin{alignedat}{6}
                    && x_1 &\ = &\   25 &\ + &\ \frac{1}{40}x_4 &\ - &\ \frac{5}{4}x_3 \\
                    && x_2 &\ = &\   75 &\ - &\ \frac{1}{40}x_4 &\ + &\ \frac{1}{4}x_3 \\
                    &\rlap{\rule{4.8cm}{.1pt}} &&&&&&&\\
                    && z   &\ = &\ 6250 &\ - &\  \frac{3}{4}x_4 &\ - &\ \frac{65}{2}x_3
                    \end{alignedat}
                    \end{align*}
                    Hier terminiert der Simplex-Algorithmus. Die optimale Lösung ist also $x_1 = 25$, $x_2 = 75$, was mit der angegebenen
                    Lösung im Skript übereinstimmt.

                \item[(ii)]
                    Zunächst bilden wir das duale Problem (D):
                    \begin{align*}
                    \begin{alignedat}{6}
                    & \text{minimiere } & 100y_1 &\ + &\ 4000y_2 & & \\
                    & \rlap{unter den Nebenbedingungen} & & & & & \\
                    && y_1 &\ + &\ 10y_2 &\ \geq &\  40\ \\
                    && y_1 &\ + &\ 40y_2 &\ \geq &\  70\ \\
                    && & & \llap{$y_1,y_2$} &\ \geq &\ 0
                    \end{alignedat}
                    \end{align*}
                    Angenommen, die eben ermittelte und im Skript angegebene Lösung ist optimal: Die beiden Ungleichungen des dualen Problems
                    müssen tatsächlich mit Gleichheit erfüllt sein, da beide Variablen der angegebenen Lösung ungleich Null sind.
                    Somit kommt man leicht auf die vermeintlich optimale Lösung von (D). Man zieht einfach die erste Gleichung von der zweiten ab,
                    und erhält trivialer Weise die Lösung: $y_1 = \frac{65}{2}$, $y_2 = \frac{3}{4}$.
                    Diese Lösung ist zulässig, und stimmt mit allen Schlupfbedingungen überein. Wir prüfen zuletzt noch den Dualitätssatz. Aus (i)
                    wissen wir, dass das Hauptproblem von (P) einen Wert von 6250 annimmt. Dies muss auch mit unserer Lösung von (D) passieren:
                    $100 \cdot \frac{65}{2} + 4000 \cdot \frac{3}{4} = 6250$. Die Lösungen sind also optimal!

            \end{enumerate}
            \item[b)]
                \underline{Starttableau}:
                \begin{align*}
                \begin{alignedat}{6}
                && x_3 &\ = &\  100 &    &    &\ - &\   x_1 &\ - &\   x_2 \\
                && x_4 &\ = &\ 4000 &\ + &\ t &\ - &\ 10x_1 &\ - &\ 50x_2 \\
                &\rlap{\rule{5.5cm}{.1pt}} &&&&&&&&&\\
                && z   &\ = &       &    &    &    &\ 40x_1 &\ + &\ 70x_2
                \end{alignedat}
                \end{align*}

                Erste Iteration:\\
                Eingangsvariable: $x_2$\\
                Ausgangsvariable: $x_4$, da $x_2 \leq 80 + \frac{1}{50}t$ eine stärkere Einschränkung ist als $x_2 \leq 100$\\
                Hier wurde genutzt, dass $t \leq 1000$
                \begin{align*}
                \begin{alignedat}{6}
                && x_2 &\ = &\   80 &\ + &\ \frac{1}{50}t &\ - &\ \frac{1}{5}x_1 &\ - &\ \frac{1}{50}x_4 \\
                && x_3 &\ = &\   20 &\ - &\ \frac{1}{50}t &\ - &\ \frac{4}{5}x_1 &\ + &\ \frac{1}{50}x_4 \\
                &\rlap{\rule{5.8cm}{.1pt}} &&&&&&&&&\\
                && z   &\ = &\ 5600 &\ + &\ \frac{7}{5}t &\ + &\ 26x_1 &\ - &\ \frac{7}{5}x_4
                \end{alignedat}
                \end{align*}

                Zweite Iteration:\\
                Eingangsvariable: $x_1$\\
                Ausgangsvariable: $x_3$, da $x_1 \leq 25 - \frac{1}{40}t$ eine stärkere Einschränkung ist als $x_1 \leq 400 + \frac{1}{10}t$\\
                Hier wurde genutzt, dass $t \geq 0$
                \begin{align*}
                \begin{alignedat}{6}
                && x_1 &\ = &\   25 &\ - &\ \frac{1}{40}t &\ + &\ \frac{1}{40}x_4 &\ - &\ \frac{5}{4}x_3 \\
                && x_2 &\ = &\   75 &\ + &\ \frac{1}{40}t &\ - &\ \frac{1}{40}x_4 &\ + &\ \frac{1}{4}x_3 \\
                &\rlap{\rule{5.8cm}{.1pt}} &&&&&&&&&\\
                && z   &\ = &\ 6250 &\ + &\ \frac{3}{4}t &\ - &\  \frac{3}{4}x_4 &\ - &\ \frac{65}{2}x_3
                \end{alignedat}
                \end{align*}

            \end{enumerate}

    \end{enumerate}
\end{document}
