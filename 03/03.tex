\newcommand{\authorinfotitle}{Carolin Konietzny, 6523939, Gruppe 3\\ Tronje Krabbe, 6435002, Gruppe 7 \\ Julian Tobergte, 6414935, Gruppe 5}
\newcommand{\authorinfo}{Carolin Konietzny, Tronje Krabbe, Julian Tobergte}
\newcommand{\titleinfo}{Optimierung 02 27.10.2014}
\newcommand{\qed}{\square}

\documentclass [a4paper,11pt]{article}
\usepackage[german,ngerman]{babel}
\usepackage[utf8]{inputenc}
\usepackage[T1]{fontenc}
\usepackage{lmodern}
\usepackage{amssymb}
\usepackage{mathtools}
\usepackage{amsmath}
\usepackage{enumerate}
%\usepackage{breqn}
\usepackage{fancyhdr}
\usepackage{multicol}

\usepackage[a4paper,left=2cm,width=13cm,right=3cm]{geometry}

% for dem plots
\usepackage{pgfplots}
%\pgfplotsset{compat=1.10}
\usepgfplotslibrary{fillbetween}
% ---

\author{\authorinfotitle}
\title{\titleinfo}
\date{\today}

\pagestyle{fancy}
\fancyhf{}
\fancyhead[R]{\authorinfo}
\fancyhead[L]{Opti Hausaufgaben}
\fancyfoot[C]{\thepage}

\begin{document}
\maketitle
    \begin{enumerate}
        % Aufgabe 1
        \item[\textbf{1.}]
        \begin{enumerate}
            \item[a)]
                \underline{Starttableau}:
                \begin{align*}
                \begin{alignedat}{6}
                & x_3 &\ = &\ \frac{1}{2} &\ - &\  x_1 &\ + &\           3x_2 \\
                & x_4 &\ = &\           3 &\ - &\  x_1 &\ + &\            x_2 \\
                & x_5 &\ = &\           1 &\ + &\ 2x_1 &\ - &\ \frac{1}{3}x_2 \\
                &\rlap{\rule{5cm}{.1pt}} &&&&&&\\
                & z   &\ = &              &    &\  x_1 &\ + &\  4x_2
                \end{alignedat}
                \end{align*}
                1. Iteration:\\
                Eingangsvariable: $x_2$\\
                Ausgangsvariable: $x_5$, was leicht ersichtlich ist, da nur in der $x_5$-Gleichung der $x_2$-Faktor negativ ist.\\
                Es folgt:
                \begin{align*}
                    & x_2 = 3 + 6x_1 - 3x_5\\
                    & x_3 = \frac{1}{2} - x_1 + 3 \left( 3 + 6x_1 - 3x_5 \right)\\
                    &   \ = \frac{19}{2} +17x_1 - 9x_5\\
                    & x_4 = 3 - x_1 \left( 3 + 6x_1 - 3x_5 \right)\\
                    &   \ = 6 + 5x_1 - 3x_5\\
                    &   z = x_1 + 4 \left( 3 + 6x_1 - 3x_5 \right)\\
                    &   \ = 12 + 25_x1 - 12x_5
                \end{align*}
                \underline{Ergebnis der 1. Iteration:}
                \begin{align*}
                \begin{alignedat}{6}
                & x_2 &\ = &\            3 &\ + &\  6x_1 &\ - &\ 3x_5 \\
                & x_3 &\ = &\ \frac{19}{2} &\ + &\ 17x_1 &\ - &\ 9x_5 \\
                & x_4 &\ = &\            6 &\ + &\  5x_1 &\ - &\ 3x_5 \\
                &\rlap{\rule{5cm}{.1pt}} &&&&&&\\
                & z   &\ = &\           12 &\ + &\ 25x_1 &\ - &\ 12x_5
                \end{alignedat}
                \end{align*}
                Die nächste Eingangsvariable wäre $x_1$, doch $x_1$ ist in keiner Schlupfvariablengleichung beschränkt.
                Also ist das Ergebnis ``unbeschränkt''.\\
                Sei $x_1 = t$, $x_2 = 3+ 6t$, $x_5 = 0$.
                So ist die entsprechende Halbgerade, in Parameterform:
                \begin{align*}
                \begin{pmatrix}
                x_1\\
                x_2
                \end{pmatrix}
                =
                \begin{pmatrix}
                0\\
                3
                \end{pmatrix}
                + t \cdot
                \begin{pmatrix}
                1\\
                6
                \end{pmatrix}
                \end{align*}
            \item[b)]
            Folgender Graph ergibt sich:

            \begin{tikzpicture}
                        \path[fill = gray!30] (0,0)--(0,3)--(0.8,8)--(8,8)--(8,5)--(4.2,1.2)--(0.5,0)--cycle;
                        \begin{axis}[
                            ymin = 0,ymax = 8,
                            xmin = 0,xmax = 8,
                            x = 1cm, y = 1cm,
                            axis x line=middle,
                            axis y line=middle,
                            axis line style=->,
                            xlabel={$x_1$},
                            ylabel={$x_2$},
                            ]
                            \addplot[domain=0:8, very thick, no marks, black] {(1/3)*x - (1/6)}; %\addlegendentry{NB 1};
                            \addplot[domain=0:8, very thick, no marks, black] {x - 3}; %\addlegendentry{NB 2};
                            \addplot[domain=0:8, very thick, no marks, black, dotted] {6 * x + 3}; %\addlegendentry{NB 3};
                            \addplot[domain=0:8, very thick, no marks, black, dotted] {-(1/4)*x + 5}; %\addlegendentry{Lösungsgerade}
                        \end{axis}
            \end{tikzpicture} \\
            Die gepunktete Halbgerade ist diejenige, die wir in Unteraufgabe a) ermittelt haben. Würde man das graphische Verfahren auf das
            LP-Problem aus Aufgabe a) anwenden, würde man die Lösungsgerade nicht einzeichnen können, da der Lösungsbereich unendlich groß ist.

            \item[c)]
            Zunächst machen wir uns klar, dass aus Unteraufgabe a) hervorgeht, dass: $z = 12 + 25t$.
            Zulässige Lösung für $z = 50$:
            \begin{align*}
                50 &= 12 + 25t\\
                38 &= 25t\\
                \frac{38}{25} &= t\\
                \Rightarrow
                    x_1 &= \frac{38}{25}\\
                \Rightarrow
                    x_2 &= \frac{303}{25}
            \end{align*}
            Zulässige Lösung für $z = 200$:
            \begin{align*}
                200 &= 12 + 25t\\
                t &= \frac{188}{25}\\
                \Rightarrow x_1 &= \frac{188}{25}\\
                \Rightarrow x_2 &= \frac{1203}{25}
            \end{align*}
            Zulässige Lösung für $z = 1000$:
            \begin{align*}
                1000 &= 12 + 25t\\
                t &= \frac{988}{25}\\
                \Rightarrow x_1 &= \frac{988}{25}\\
                \Rightarrow x_2 &= \frac{6003}{25}
            \end{align*}
        \end{enumerate}
        % Aufgabe 2
        \item[\textbf{2.}]
            
    \end{enumerate}


\end{document}
