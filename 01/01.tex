\newcommand{\authorinfotitle}{Tronje Krabbe, 6435002, Julian Tobergte, 6414935}
\newcommand{\authorinfo}{Tronje Krabbe, Julian Tobergte}
\newcommand{\titleinfo}{Optimierung 01 20.10.2014}
\newcommand{\qed}{\square}

\documentclass [a4paper,11pt]{article}
\usepackage[german,ngerman]{babel}
\usepackage[utf8]{inputenc}
\usepackage[T1]{fontenc}
\usepackage{lmodern}
\usepackage{amssymb}
\usepackage{mathtools}
\usepackage{amsmath}
\usepackage{enumerate}
%\usepackage{breqn}
\usepackage{fancyhdr}
\usepackage{multicol}

% for dem plots
\usepackage{pgfplots}
%\pgfplotsset{compat=1.10}
\usepgfplotslibrary{fillbetween}
% ---

\author{\authorinfotitle}
\title{\titleinfo}
\date{\today}

\pagestyle{fancy}
\fancyhf{}
\fancyhead[R]{\authorinfo}
\fancyhead[L]{Optimierung Hausaufgaben}
\fancyfoot[C]{\thepage}

\begin{document}
\maketitle
    \begin{enumerate}
        % Aufgabe 1
        \item[\textbf{1.}]
            \begin{enumerate}
                \item[a)]
                    Wir formulieren die gegebenen Probleme in Standardform:
                    \begin{enumerate}
                        \item[(i)]
                            \begin{align*}
                            \begin{alignedat}{6}
                            & \text{maximiere } & -x_1 &\ + &\ x_2 &\ + &\ x_3 &\ - &\ 2x_4 & & \\
                            & \rlap{unter den Nebenbedingungen} & & & & & & & & & \\
                            && 7x_1 &\ - &\  x_2 &\ + &\  x_3 &    &       &\ \leq &\ 2 \\
                            &&      &\ - &\ 5x_2 &\ + &\  x_3 &\ - &\  x_4 &\ \leq &\ 7 \\
                            &&      &\  &\ 5x_2 &\ - &\  x_3 &\ + &\  x_4 &\ \leq &\ -7 \\
                            && -3x_1 &\ + &\  x_2 &\ + &\ 2x_3 &\ - &\  x_4 &\ \geq &\ -3 \\
                            && & & & & & & \llap{$x_1, x_2, x_3,x_4$} &\ \geq &\ 0
                            \end{alignedat}
                            \end{align*}
                        \item[(ii)]
                            % \begin{align*}
                            %     \text{Maximiere } x_1 - x_2 - x_3 +2x_4\\
                            %     \text{unter den Nebenbedingungen}\\
                            %     -7x_1 + x_2 + 4x_3' - 4x_3'' \leq -2\\
                            %     3x_1 - x_2 - 2x_3' + x_3'' + x_4 \leq 3\\
                            %     x_2 - 2x_4 \leq 7\\
                            %     -x_2 + 2x_4 \leq -7\\
                            %     x_4 \leq -9\\
                            %     x_1,x_2,x_3',x_3'',x_4 \geq 0
                            % \end{align*}
                            \begin{align*}
                            \begin{alignedat}{7}
                            & \text{maximiere } & x_1 &\ - &\ x_2 &\ - &\ x_3' &\ + &\ x_3'' &\ + &\ 2x_4 & & \\
                            & \rlap{unter den Nebenbedingungen} & & & & & & & & & & & \\
                            &&-7x_1 &\ + &\ x_2 &\ + &\ 4x_3' &\ - &\ 4x_3'' &    &       &\ \leq &\  -2 \\
                            && 3x_1 &\ - &\ x_2 &  - &\ 2x_3' &\ + &\ x_3''  &\ + &\  x_4 &\ \leq &\  3 \\
                            &&      &    &\ x_2 &    &        &    &         &\ - &\ 2x_4 &\ \leq &\  7 \\
                            &&      &\ - &\ x_2 &    &        &    &         &\ + &\ 2x_4 &\ \leq &\  -7 \\
                            &&      &    &      &    &        &    &         &    &\ x_4  &\ \leq &\  -9 \\
                            &&      &    &      &    &        &    &         &    & \llap{$x_1, x_2, x_3', x_3'', x_4$} &\ \geq &\ 0
                            \end{alignedat}
                            \end{align*}
                    \end{enumerate}
                \item[b)]
%<<<<<<< HEAD

                    \begin{tikzpicture}
                        \path[fill = gray!30] (0,0)--(0,2.5)--(9/4,19/4)--(24/7,25/7)--(2,0)--cycle;
                        \begin{axis}[
                            ymin = 0,ymax = 8.5,
                            xmin = 0,xmax = 8.5,
                            x = 1cm, y = 1cm,
                            axis x line=middle,
                            axis y line=middle,
                            axis line style=->,
                            xlabel={$x_1$},
                            ylabel={$x_2$},
                            ]
                            \addplot[name path=a,domain=0:8, very thick, no marks, black!50] {(5/2)*x - 5}; %\addlegendentry{NB 1};
                            \addplot[name path=b,domain=0:8, very thick, no marks, black!50] {x + (5/2)}; %\addlegendentry{NB 2};
                            \addplot[name path=c,domain=0:8, very thick, no marks, black!50] {-x +7}; %\addlegendentry{NB 3};
                            \addplot[domain=0:8, very thick, no marks, black, dotted] {-(5/3) * x + 28/3}; %\addlegendentry{Lösungsgerade}
                        \end{axis}
                    \end{tikzpicture}
                    \\
                    Die optimale Lösung ist also bei $x_1 = \frac{24}{7}$ und $x_2 = \frac{25}{7}$
%=======
                % I have no idea what I'm doing here
%                 \begin{tikzpicture} 
%     				\begin{axis}[enlargelimits=0]
%     				\addplot[name path=f,domain=0:15,blue] {-5x/2 + 5};
% 					%main function    				
%     				\addplot[name path=m,domain=0:15,black] {-5x/3};

%     				\path[name path=axis] (axis cs:0,0) -- (axis cs:1,0);

%    					\addplot [
%         				thick,
%         				color=blue,
%         				fill=blue, 
%         				fill opacity=0.3
%     				]
%     				fill between[
%         			of=f and axis,
%         			soft clip={domain=0:15},
%     				];
% 					% no use		
%     				%\node [rotate=48] at (axis cs:  .7,  .59) {$y=x^2$};
%     				%\node [rotate=90] at (axis cs:  1.05,  .25) {$x=1$};
%     				\end{axis}
% 				\end{tikzpicture}  
% %>>>>>>> 695ead62eb3eef60fed23d16fa365c493cf43b63

            \end{enumerate}
        % Aufgabe 2
        \item[\textbf{2.}]
        \begin{enumerate}
                \item[a)]
                    Wir formulieren die gegebenen Probleme in Standardform. Sei (je 100 gramm)\\
                    $x_1$ = Weißbrot, $x_2$ = Käse,\\
                    $x_3$ = Hähnchen, $x_4$ = Fisch,\\
                    $x_5$ = Backpflaumen, $x_6$ = Nüsse,\\
                    $x_7$ = Schwarzbrot, $x_8$ = Margarine\\

                \begin{align*}
                \begin{alignedat}{10}
                & \text{maximiere } & -67x_1 &\ - &\ 120x_2 &\ - &\ 100x_3 &\ - &\ 90x_4 &\ - &\ 97x_5 &\ - &\ 124x_6 &\ - &\ 98x_7 &\ - &\ 62x_8 & & \\
                & \rlap{unter den Nebenbedingungen} & & & & & & & & & & & & & & & & & \\
                &&  8x_1 &\ + &\ 25x_2 &\ + &\ 30x_3 &\ + &\ 22x_4 &\ + &\  3x_5 &\ + &\  8x_6 &\ + &\  6x_7 &     &        &\ \leq &\  75 \\
                &&   x_1 &\ + &\ 35x_2 &\ + &\  8x_3 &\ + &\   x_4 &    &        &\ + &\ 33x_6 &\ + &\ 13x_7 &\ +  &\ 98x_8 &\ \leq &\  90 \\
                && 54x_1 &    &        &    &        &    &        &\ + &\ 42x_5 &\ + &\  4x_6 &\ + &\ 63x_7 &     &        &\ \leq &\ 300 \\
                &&       &    &        &    &        &    &        &    &        &    &        &\ - &\   x_7 &     &        &\ \leq &\ -0.8 \\
                && & & & & & & & & & & & & & & \llap{$x_1, x_2, x_3, x_4, x_5, x_6, x_7, x_8$} &\ \geq &\ 0
                \end{alignedat}
                \end{align*}

				\item[b)]
					Wir formulieren die gegebenen Probleme in Standardform.
                    Sei (je 100 gramm) $x_1$ = Tomate, $x_2$ = Kopfsalat, $x_3$ = Spinat, $x_4$ = Möhren, $x_5$ = Öl
       %                      \begin{align*}
                            
 						% 		\text{Maximiere } -21x_1 -16x_2 -371x_3 -346x_4 -884x_5\\
							% 		\text{unter den Nebenbedingungen}\\
							% 		0.85x_1 + 1.62x_2 + 12.78x_3 + 8.39x_4 \leq 15\\
							% 		0.33x_1 + 0.2x_2 + 1.58x_3 + 1.39x_4 + 100x5 \leq 2\\
							% 		-0.33x_1 - 0.2x_2 - 1.58x_3 - 1.39x_4 - 100x5 \leq -6\\
							% 		4.64x_1 + 2.37x_2 + 74.69x_3 + 80.70x_4 \leq 4\\
							% 		-9x_1 - 8x_2 - 7x_3 - 508.2x_4 \leq -0.1\\
							% 		- x_2 - x_3 + x_1 + x_4 + x_5 \leq 0\\
							% 		x_1,x_2,x_3,x_4,x_5,x_6,x_8 \geq 0						
							% \end{align*}


                    \begin{align*}
                    \begin{alignedat}{7}
                    & \text{maximiere } & -21x_1 &\ - &\ 16x_2 &\ - &\ 371x_3 &\ - &\ 346x_4 &\ - &\ 884x_5 & & \\
                    & \rlap{unter den Nebenbedingungen} & & & & & & & & & & & \\
                    &&-0.85x_1 &\ - &\ 1.62x_2 &\ - &\ 12.78x_3 &\ - &\  8.39x_4 &    &         &\ \leq &\ -15 \\
                    &&-0.33x_1 &\ - &\  0.2x_2 &\ - &\  1.58x_3 &\ - &\  1.39x_4 &\ - &\ 100x_5 &\ \leq &\  -2 \\
                    && 0.33x_1 &\ + &\  0.2x_2 &\ + &\  1.58x_3 &\ + &\  1.39x_4 &\ + &\ 100x_5 &\ \leq &\   6 \\
                    &&-4.64x_1 &\ - &\ 2.37x_2 &\ - &\ 74.69x_3 &\ - &\ 80.70x_4 &    &         &\ \leq &\  -4 \\
                    &&    9x_1 &\ + &\    8x_2 &\ + &\     7x_3 &\ + &\ 508.2x_4 &    &         &\ \leq &\ 0.1 \\
                    &&    -x_1 &\ + &\     x_2 &\ + &\      x_3 &\ - &\      x_4 &\ - &\    x_5 &\ \leq &\   0 \\
                    && & & & & & & & & \llap{$x_1, x_2, x_3, x_4, x_5$} &\ \geq &\ 0
                    \end{alignedat}
                    \end{align*}
		\end{enumerate}
        % Aufgabe 3
        %\item[\textbf{3.}]
% we're not quite there yet
        % Aufgabe 4
        %\item[\textbf{4.}]
            
            

    \end{enumerate}


\end{document}