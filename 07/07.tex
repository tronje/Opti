\newcommand{\authorinfotitle}{Carolin Konietzny, 6523939, Gruppe 3\\ Tronje Krabbe, 6435002, Gruppe 7 \\ Julian Tobergte, 6414935, Gruppe 5}
\newcommand{\authorinfo}{Carolin Konietzny, Julian Tobergte, Tronje Krabbe}
\newcommand{\titleinfo}{Optimierung 07}
\newcommand{\qed}{\square}
\newcommand{\tilD}{($\widetilde{\mathrm{D}}$)}

\documentclass [a4paper,11pt]{article}
\usepackage[german,ngerman]{babel}
\usepackage[utf8]{inputenc}
\usepackage[T1]{fontenc}
\usepackage{lmodern}
\usepackage{amssymb}
\usepackage{mathtools}
\usepackage{amsmath}
\usepackage{enumerate}
%\usepackage{breqn}
\usepackage{fancyhdr}
\usepackage{multicol}
\usepackage{eurosym}

\usepackage[a4paper,left=2cm,width=13cm,right=3cm]{geometry}

% for dem plots
\usepackage{pgfplots}
%\pgfplotsset{compat=1.10}
\usepgfplotslibrary{fillbetween}
% ---

\author{\authorinfotitle}
\title{\titleinfo}
\date{\today}

\pagestyle{fancy}
\fancyhf{}
\fancyhead[R]{\authorinfo}
\fancyhead[L]{Opti Hausaufgaben}
\fancyfoot[C]{\thepage}

\begin{document}
\maketitle
    \begin{enumerate}
        % Aufgabe 1
        \item[\textbf{1.}]
            \begin{enumerate}
                \item[a)]
                    Zunächst stellen wir das duale Problem (D) auf:
                    \begin{align*}
                    \begin{alignedat}{6}
                    & \text{minimiere } & \frac{22}{3}y_1 &\ + &\ \frac{4}{3}y_2 &\ + &\ \frac{13}{3}y_3 & & \\
                    & \rlap{unter den Nebenbedingungen} & & & & & & & \\
                    &&  y_1 &\ + &\ 4y_2 &\ + &\ 2y_3 &\ \geq &\ 12\ \\
                    && 3y_1 &\ + &\ 2y_2 &\ + &\  y_3 &\ \geq &\ 11\ \\
                    && 2y_1 &\ + &\  y_2 &\ + &\ 5y_3 &\ \geq &\  7\ \\
                    && 5y_1 &\ - &\ 2y_2 &\ + &\ 4y_3 &\ \geq &\  5\ \\
                    && & & & & \llap{$y_1,y_2,y_3$} &\ \geq &\ 0
                    \end{alignedat}
                    \end{align*}
                    Für folgende Aussagen nehmen wir an, dass die vorgeschlagene Lösung optimal ist.\\
                    Durch einsetzen der vorgeschlagenen Lösung in das primale Problem wissen wir, dass $y_3$ gleich Null sein muss. Dies ist trivialer Weise
                    daraus zu folgern, dass, nach Einsetzen, die erste und zweite Ungleichung von (P) mit Gleichheit, die dritte jedoch mit Ungleichheit erfüllt
                    sind. Außerdem wissen wir, dass in (D) die zweite und vierte Ungleichung mit Gleichheit erfüllt sind, da die entsprechenden Variablen der
                    vorgeschlagenen Lösung nicht gleich Null sind. Es bleibt folgendes Gleichungssystem zu lösen:
                    \begin{align*}
                    3y_1 + 2y_2 &= 11\ \\
                    5y_1 - 2y_2 &= 5\
                    \end{align*}
                    Wir addieren beide Gleichungen und erhalten $8y_1 = 16 \Rightarrow y_1 = 2$. Es folgt trivialer Weise: $y_2 = \frac{5}{2}$.\\
                    Die Lösung für (D) ist also:
                    \begin{align*}
                        y_1 &= 2\\
                        y_2 &= \frac{5}{2}\\
                        y_3 &= 0
                    \end{align*}
                    Wir überprüfen zunächst, ob der Dualitätssatz erfüllt ist:
                    \begin{align*}
                        11 \cdot \frac{4}{3} + 5 \cdot \frac{2}{3} = \frac{54}{3}\\
                        \frac{22}{3} \cdot 2 + \frac{4}{3} \cdot \frac{5}{2} = \frac{54}{3}
                    \end{align*}
                    Der Dualitätssatz ist also erfüllt.
                    Als nächstes überprüfen wir die Nebenbedingungen, und stellen fest, dass die dritte mit unserer Lösung gar nicht erfüllt ist:
                    \begin{align*}
                        2 \cdot 2 + \frac{5}{2} = \frac{13}{2} \leq 7
                    \end{align*}
                    Somit ist die duale ``Lösung'' gar nicht zulässig, und die vorgeschlagene Lösung nicht optimal. $\qed$
                \item[b)]
                    Wir stellen wieder zunächst das duale Problem (D) auf:
                    \begin{align*}
                    \begin{alignedat}{6}
                    & \text{minimiere } & 5y_1 &\ + &\ 2y_2 &\ + &\ 2y_3 & & \\
                    & \rlap{unter den Nebenbedingungen} & & & & & & & \\
                    && 2y_1 &\ - &\  y_2 &    &       &\ \geq &\  1\ \\
                    &&      &    &\ 3y_2 &\ + &\  y_3 &\ \geq &\  6\ \\
                    &&  y_1 &\ - &\ 2y_2 &\ - &\  y_3 &\ \geq &\ -4\ \\
                    && & & & & \llap{$y_1,y_2,y_3$} &\ \geq &\ 0
                    \end{alignedat}
                    \end{align*}
                    Wir nehmen an, die vorgeschlagene Lösung sei optimal: \\
                    Auch hier ist, wie in a), $y_3 = 0$, da die dritte Ungleichung des primalen Problems nicht durch Gleichheit erfüllt ist. Aus der
                    vorgeschlagenen Lösung erkennen wir, dass die erste und zweite Ungleichung von (D) durch Gleichheit erfüllt sein müssen.
                    Aufgrund dessen können wir aus der zweiten Nebenbedingung direkt ablesen: $y_2 = 2$. Daraus folgt trivialer Weise: $y_1 = \frac{3}{2}$.
                    Wir fassen zusammen:
                    \begin{align*}
                    y_1 &= \frac{3}{2}\\
                    y_2 &= 2\\
                    y_3 &= 0
                    \end{align*}
                    Durch einsetzen dieser vermeintlich optimalen Lösung in (D) sehen wir, dass keine Schlupfbedingungen verletzt werden. Das heißt, die
                    erste und zweite Nebenbedingung sind durch Gleichheit erfüllt, die dritte durch Ungleichheit. Setzen wir diese und die vorgeschlagene
                    Lösung in ihre entsprechenden Hauptprobleme ein, erhalten wir bei beiden als Ergebnis $\frac{23}{2}$. Dies bedeutet, nach dem Satz vom
                    komplementären Schlupf und dem Dualitätssatz, dass die vorgeschlagene Lösung optimal ist. $\qed$

            \end{enumerate}

        % Aufgabe 2
        \item[\textbf{2.}]
            \begin{enumerate}
                \item[a)]
                    Wir rechnen:\\
                    $ 10 \cdot 15 + 105 \cdot 11 + 20 \cdot 13 + 220 \cdot 10 + 200 \cdot 14 = 6565 $
                \item[b)]
                    Wir entnehmen das LP-Problem zu der entsprechenden Aufgabe von Blatt 5 aus der in Stine bereitgestellten Musterlösung:\\
                    \begin{align*}
                    \begin{alignedat}{10}
                    & \text{maximiere } & 8x_1 &\ + &\ 3x_2 &\ + &\ 9x_3 &\ + &\ 5x_4 &\ + &\ 6x_5 &\ + &\ 3x_6 & & \\
                    & \rlap{unter den Nebenbedingungen} & & & & & & & & & & & & & \\
                    && x_1 &\ + &\ x_2 & & & & & & & & &\ \leq &\ 200\ \\
                    &&     &    &      & & x_3 &\ + &\ x_4 & & & & &\ \leq &\ 115\ \\
                    &&     &    &      & &     &    &      & &\ x_5 &\ + &\ x_6 &\ \leq &\ 240\ \\
                    && x_1 &    &      &\ + &\ x_3 & & &\ + &\ x_5 & & &\ \leq &\ 210\ \\
                    &&     &    &\ x_2 &    &      &\ + &\ x_4 & & &\ + &\ x_6 &\ \leq &\ 125\ \\
                    && & & & & & & & & & & \llap{$x_1,x_2,x_3,x_4,x_5,x_6$} &\ \geq &\ 0
                    \end{alignedat}
                    \end{align*}
                    Wir stellen dazu das duale Problem (D) auf:
                    \begin{align*}
                    \begin{alignedat}{10}
                    & \text{minimiere } & 200y_1 &\ + &\ 115y_2 &\ + &\ 240y_3 &\ + &\ 210y_4 &\ + &\ 125y_5 & & \\
                    & \rlap{unter den Nebenbedingungen} & & & & & & & & & & &  \\
                    && y_1 & & & & &\ + &\ y_4 & & &\ \geq &\ 8\ \\
                    && y_1 & & & & & & &\ + &\ y_5 &\ \geq &\ 3\ \\
                    &&  & &\ y_2 & & &\ + &\ y_4 & & &\ \geq &\ 9\ \\
                    &&  & &\ y_2 & & & & &\ + &\ y_5 &\ \geq &\ 5\ \\
                    && & & & &\ y_3 &\ + &\ y_4 & & &\ \geq &\ 6\ \\
                    && & & & &\ y_3 & & &\ + &\ y_5 &\ \geq &\ 3\ \\
                    && & & & & & & & & \llap{$y_1,y_2,y_3,y_4,y_5,$} &\ \geq &\ 0
                    \end{alignedat}
                    \end{align*}
                    Wir nehmen an, die vorgeschlagene Lösung sei optimal. Sie lautet:
                    \begin{align*}
                        x_1 &= 200\\
                        x_2 &= 0\\
                        x_3 &= 10\\
                        x_4 &= 105\\
                        x_5 &= 0\\
                        x_6 &= 20
                    \end{align*}
                    Hieraus folgt, dass alle Ungleichungen von (D), mit Ausnahme der zweiten und fünften, mit Gleichheit erfüllt sein müssen. Setzen wir die
                    Lösung in (P) ein, stellen wir fest, dass alle Ungleichungen mit Gleichheit erfüllt sind, bis auf die dritte. Wir wissen also schonmal:
                    $y_3 = 0$. Es bleibt folgendes Gleichungssystem zu lösen:
                    \begin{align*}
                    \begin{alignedat}{10}
                    && y_1 &    &     & & &\ + &\ y_4 & & &\ = &\ 8\ \\
                    &&     &    &\ y_2 & & &\ + &\ y_4 & & &\ = &\ 9\ \\
                    &&     &    &\ y_2 & & & & &\ + &\ y_5 &\ = &\ 5\ \\
                    &&     &    & & &\ y_3 & & &\ + &\ y_5 &\ = &\ 3\ 
                    \end{alignedat}
                    \end{align*}
                    Da $y_3 = 0$ ist die Lösung trivial. Man sieht sofort, dass $y_5 = 3$, und kann anhand dessen leicht die Werte der anderen Variablen erkennen:
                    \begin{align*}
                        y_1 &= 1\\
                        y_2 &= 2\\
                        y_3 &= 0\\
                        y_4 &= 7\\
                        y_5 &= 3
                    \end{align*}
                    Die Schlupfbedingungen werden von dieser Lösung nicht verletzt, das heißt, es herscht überall Ungleichheit, wo Ungleichheit herschen sollte.
                    Nun prüfen wir, ob der Dualitätssatz erfüllt ist, oder nicht. Setzt man die vorgeschlagene Lösung in die Hauptbedingung des primalen Problems
                    ein, erhält man als Ergebnis $2275$. Setzt man die eben erhaltene Lösung des dualen Problems in die Hauptbedingung des selben ein,
                    erhält man als Ergebnis ebenfalls $2275$. Die vorgeschlagene Lösung ist dementsprechend optimal! $\qed$
            \end{enumerate}

    \end{enumerate}
\end{document}
