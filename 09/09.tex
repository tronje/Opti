\newcommand{\authorinfotitle}{Carolin Konietzny, 6523939, Gruppe 3\\ Tronje Krabbe, 6435002, Gruppe 7 \\ Julian Tobergte, 6414935, Gruppe 5}
\newcommand{\authorinfo}{Carolin Konietzny, Julian Tobergte, Tronje Krabbe}
\newcommand{\titleinfo}{Optimierung 09}
\newcommand{\qed}{\square}
\newcommand{\tilD}{($\widetilde{\mathrm{D}}$)}

\documentclass [a4paper,11pt]{article}
\usepackage[german,ngerman]{babel}
\usepackage[utf8]{inputenc}
\usepackage[T1]{fontenc}
\usepackage{lmodern}
\usepackage{amssymb}
\usepackage{mathtools}
\usepackage{amsmath}
\usepackage{enumerate}
%\usepackage{breqn}
\usepackage{fancyhdr}
\usepackage{multicol}
\usepackage{eurosym}

\usepackage[a4paper,left=2cm,width=13cm,right=3cm]{geometry}

% for dem plots
\usepackage{pgfplots}
%\pgfplotsset{compat=1.10}
\usepgfplotslibrary{fillbetween}
% ---

\author{\authorinfotitle}
\title{\titleinfo}
\date{\today}

\pagestyle{fancy}
\fancyhf{}
\fancyhead[R]{\authorinfo}
\fancyhead[L]{Opti Hausaufgaben}
\fancyfoot[C]{\thepage}

\begin{document}
\maketitle
    \begin{enumerate}
        % Aufgabe 1
        \item[\textbf{1.}]
        \begin{enumerate}
        \item[a)]
            Die Eingangsdaten sind gegeben durch
            \begin{align*}
                A =
                \stackrel{\mbox{$ \ x_1\ \ x_2\ \ x_3\ \ x_4\ \ x_5$}}{%
                    \begin{pmatrix}
                        -1 & 2 & 1 & 0 & 0\\
                         0 & 1 & 0 & 1 & 0\\
                         1 & 0 & 0 & 0 & 1
                    \end{pmatrix}
                  },\ b = 
                \begin{pmatrix}
                    3\\
                    2\\
                    5
                \end{pmatrix}\ \text{und}\ c^T =
                \stackrel{\mbox{$x_1\ x_2\ x_3\ x_4\ x_5$}}{
                    \begin{pmatrix}
                       2 & 1 & 0 & 0 & 0
                    \end{pmatrix}\text{.}
                }
            \end{align*}
            Das Verfahren startet mit
            \begin{align*}
                x^*_B =
                \begin{pmatrix}
                    3\\
                    2\\
                    5
                \end{pmatrix}\ \text{und}\ B =
                \stackrel{\mbox{$ x_3\ x_4\ x_5 $}}{
                    \begin{pmatrix}
                        1 & 0 & 0\\
                        0 & 1 & 0\\
                        0 & 0 & 1
                    \end{pmatrix}
                }
                \text{.}
            \end{align*}

            \underline{\textbf{\textit{1. Iteration}}}\\
            \underline{\textit{1. Schritt:}}\\
            $y^T = \begin{pmatrix} 0 & 0 & 0 \end{pmatrix}$, da $B$ in dieser Iteration noch die Einheitsmatrix, und $c^T_B$ der Nullvektor ist.\\
            
            \underline{\textit{2. Schritt:}}\\
            Für alle Spalten von $A_N$ gilt $y^T a = 0$, und die entsprechenden Komponenten von $c^T$ lauten $2$ und $1$, weshalb alle Spalten von $A_N$
            als Eingangsspalte $a$ infrage kommen; wir wählen $a = \begin{pmatrix} -1\\ 0\\ 1 \end{pmatrix}$.
            Unsere Eingangsvariable ist demnach $x_1$.\\

            \underline{\textit{3. Schritt:}}\\
            Wir lösen das Gleichungssystem $Bd=a$ und erhalten, da $B$ die Einheitsmatrix ist, $d=a=\begin{pmatrix} -1\\ 0\\ 1 \end{pmatrix}$.

            \newpage

            \underline{\textit{4. Schritt:}}\\
            \begin{align*}
                \begin{pmatrix}
                    3\\
                    2\\
                    5
                \end{pmatrix}
                - t
                \begin{pmatrix}
                    -1\\
                    0\\
                    1
                \end{pmatrix}
                \geq
                \begin{pmatrix}
                    0\\
                    0\\
                    0
                \end{pmatrix}
            \end{align*}
            Wir lösen dieses Gleichungssystem und erhalten $t=5$ als größtes $t$. Daraus folgt:
            \begin{align*}
                x^*_B - td = \begin{pmatrix} 8\\ 2\\ 0 \end{pmatrix}
            \end{align*}

            \underline{\textit{5. Schritt:}}\\
            Update:
            \begin{align*}
                x^*_B =
                \begin{pmatrix}
                    x_3\\ x_4\\ x_1
                \end{pmatrix}
                =
                \begin{pmatrix}
                    8\\ 2\\ 5
                \end{pmatrix},\ B=
                \begin{pmatrix}
                    1 & 0 & -1\\
                    0 & 1 & 0\\
                    0 & 0 & 1
                \end{pmatrix}
            \end{align*}

            \underline{\textbf{\textit{2. Iteration}}}\\
            \underline{\textit{1. Schritt:}}\\
            \begin{align*}
                y^T \cdot
                \begin{pmatrix}
                    1 & 0 & -1\\
                    0 & 1 & 0\\
                    0 & 0 & 1
                \end{pmatrix} =
                \begin{pmatrix}
                    2 & 0 & 0
                \end{pmatrix}
            \end{align*}
            \underline{\textit{2. Schritt:}}\\
            Nun lässt sich keine Spalte in $A_N$ mehr finden, für die die Bedingung gilt. Die aktuelle Lösung ist also optimal und lautet:\\
            $x_1 = 5$, $x_2 = 1$ und $z = 11$.


        \item[b)]
            Die Eingangsdaten sind gegeben durch
            \begin{align*}
                A =
                \stackrel{\mbox{$ \ x_1\ \ x_2\ \ x_3\ \ x_4\ \ x_5\ \ x_6$}}{%
                    \begin{pmatrix}
                        -2 & -1 & 3 & 1 & 0 & 0\\
                         1 & 2 & 0 & 0 & 1 & 0\\
                         -1 & 0 & 1 & 0 & 0 & 1
                    \end{pmatrix}
                  },\ b = 
                \begin{pmatrix}
                    2\\
                    5\\
                    2
                \end{pmatrix}\ \text{und}\ c^T =
                \stackrel{\mbox{$x_1\ \ x_2\ \ x_3\ x_4\ \ x_5\ \ x_6$}}{
                    \begin{pmatrix}
                       -4 & 1 & 6 & 0 & 0 & 0
                    \end{pmatrix}\text{.}
                }
            \end{align*}
            Das Verfahren startet mit
            \begin{align*}
                x^*_B =
                \begin{pmatrix}
                    2\\
                    5\\
                    2
                \end{pmatrix}\ \text{und}\ B =
                \stackrel{\mbox{$ x_4\ x_5\ x_6 $}}{
                    \begin{pmatrix}
                        1 & 0 & 0\\
                        0 & 1 & 0\\
                        0 & 0 & 1
                    \end{pmatrix}
                }
                \text{.}
            \end{align*}

            \newpage

            \underline{\textbf{\textit{1. Iteration:}}}\\
            \underline{\textit{1. Schritt:}}\\
            $y^T = \begin{pmatrix} 0 & 0 & 0 \end{pmatrix}$, da $B$ in dieser Iteration noch die Einheitsmatrix, und $c^T_B$ der Nullvektor ist.\\

            \underline{\textit{2. Schritt:}}\\
            Wir können aus der $x_2$ und der $x_3$ Zeile frei wählen, und nehmen $a = \begin{pmatrix} -1\\ 2\\ 0 \end{pmatrix}$. Die Eingangsvariable
            ist also $x_2$.\\

            \underline{\textit{3. Schritt:}}\\
            $d = a = \begin{pmatrix} -1\\ 2\\ 0 \end{pmatrix}$.

            \underline{\textit{4. Schritt:}}\\
            \begin{align*}
                \begin{pmatrix}
                    2\\ 5\\ 2
                \end{pmatrix} - t
                \begin{pmatrix}
                    -1\\ 2\\ 0
                \end{pmatrix} \geq
                \begin{pmatrix}
                    0\\ 0\\ 0
                \end{pmatrix}
            \end{align*}
            Wir lösen das Gleichungssystem und erhalten $t = \frac{5}{2}$. Also:
            \begin{align*}
                \begin{pmatrix}
                    2\\ 5\\ 2
                \end{pmatrix} - \frac{5}{2}
                \begin{pmatrix}
                    -1\\ 2\\ 0
                \end{pmatrix} =
                \begin{pmatrix}
                    \frac{9}{2}\\
                    0\\
                    2
                \end{pmatrix}
            \end{align*}
            Da $x_B^* = \begin{pmatrix} x_4^*\\ x_5^*\\ x_6^* \end{pmatrix}$, ist $x_5^*$ unsere Ausgangsvariable.\\

            \underline{\textit{5. Schritt:}}\\
            Update:
            \begin{align*}
                x^*_B =
                \begin{pmatrix}
                    4.5\\
                    2.5\\
                    2
                \end{pmatrix}\ \text{und}\ B =
                \stackrel{\mbox{$ x_4\ \ x_2\ \ x_6 $}}{
                    \begin{pmatrix}
                        1 & -1 & 0\\
                        0 & 2 & 0\\
                        0 & 0 & 1
                    \end{pmatrix}
                }
            \end{align*}

            \underline{\textbf{\textit{2. Iteration:}}}\\
            \underline{\textit{1. Schritt:}}\\
            \begin{align*}
                y^T
                \begin{pmatrix}
                    1 & -1 & 0\\
                    0 & 2 & 0\\
                    0 & 0 & 1
                \end{pmatrix} = 
                \begin{pmatrix}
                    0 & 1 & 0
                \end{pmatrix}
                \Rightarrow y^T =
                \begin{pmatrix}
                    \frac{1}{2} & \frac{1}{2} & 0
                \end{pmatrix}
            \end{align*}

            \underline{\textit{2. Schritt:}}\\
            $a = \begin{pmatrix} 3\\ 0\\ 1 \end{pmatrix}$. Daraus folgt, dass unsere Eingangsvariable $x_3$ ist.\\

            \underline{\textit{3. Schritt:}}\\
            \begin{align*}
                \begin{pmatrix}
                    1 & -1 & 0\\
                    0 & 2 & 0\\
                    0 & 0 & 1
                \end{pmatrix} \cdot d =
                \begin{pmatrix}
                    3\\ 0\\ 1
                \end{pmatrix}
                \Rightarrow
                \begin{pmatrix}
                    3\\ 0\\ 1
                \end{pmatrix}
            \end{align*}

            \underline{\textit{4. Schritt:}}\\
            \begin{align*}
                \begin{pmatrix}
                    4.5\\
                    2.5\\
                    2
                \end{pmatrix} - t
                \begin{pmatrix}
                    3\\ 0\\ 1
                \end{pmatrix} \geq
                \begin{pmatrix}
                    0\\ 0\\ 0
                \end{pmatrix} \Rightarrow
                t = \frac{3}{2}
            \end{align*}
            Es folgt:
            \begin{align*}
                \begin{pmatrix}
                    4.5\\
                    2.5\\
                    2
                \end{pmatrix} - \frac{3}{2}
                \begin{pmatrix}
                    3\\ 0\\ 1
                \end{pmatrix} =
                \begin{pmatrix}
                    0\\
                    2.5\\
                    0.5
                \end{pmatrix}
            \end{align*}
            Da $x_B^* = \begin{pmatrix} x_4^*\\ x_2^*\\ x_6^* \end{pmatrix}$, ist die Ausgangsvariable $x_4$.\\

            \underline{\textit{5. Schritt:}}\\
            Update:
            \begin{align*}
                x_B^* =
                \begin{pmatrix}
                    x_3^*\\
                    x_2^*\\
                    x_6^*
                \end{pmatrix} =
                \begin{pmatrix}
                    1.5\\
                    2.5\\
                    1.5
                \end{pmatrix} \ \text{und}\
                \stackrel{\mbox{$ x_3\ \ x_2\ \ x_6 $}}{
                    \begin{pmatrix}
                        3 & -1 & 0\\
                        0 & 2 & 0\\
                        1 & 0 & 1
                    \end{pmatrix}
                }
            \end{align*}

            \underline{\textbf{\textit{3. Iteration:}}}\\
            \underline{\textit{1. Schritt:}}\\
            \begin{align*}
                y^T = 
                \begin{pmatrix}
                    3 & -1 & 0\\
                    0 & 2 & 0\\
                    1 & 0 & 1
                \end{pmatrix} =
                \begin{pmatrix}
                    6\\ 1\\ 0
                \end{pmatrix} \Rightarrow
                y^T =
                \begin{pmatrix}
                    \frac{13}{6}\\
                    0.5\\
                    -\frac{13}{6}
                \end{pmatrix}
            \end{align*}

            Hier terminiert der Algorithmus, mit der optimalen Lösung: $x_1 = 0$, $x_2 = \frac{5}{2}$, $x_3 = 1.5$ und $z = \frac{23}{2}$.

        \end{enumerate}

        % Aufgabe 2
        \item[\textbf{2.}]


    \end{enumerate}
\end{document}