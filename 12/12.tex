\newcommand{\authorinfotitle}{Carolin Konietzny, 6523939, Gruppe 3\\ Tronje Krabbe, 6435002, Gruppe 7 \\ Julian Tobergte, 6414935, Gruppe 5}
\newcommand{\authorinfo}{Carolin Konietzny, Julian Tobergte, Tronje Krabbe}
\newcommand{\titleinfo}{Optimierung 12}
\newcommand{\qed}{\square}
\newcommand{\tilD}{($\widetilde{\mathrm{D}}$)}

\documentclass [a4paper,11pt]{article}
\usepackage[german,ngerman]{babel}
\usepackage[utf8]{inputenc}
\usepackage[T1]{fontenc}
\usepackage{lmodern}
\usepackage{amssymb}
\usepackage{mathtools}
\usepackage{amsmath}
\usepackage{enumerate}
%\usepackage{breqn}
\usepackage{fancyhdr}
\usepackage{multicol}
\usepackage{eurosym}

\usepackage[a4paper,left=2cm,width=13cm,right=3cm]{geometry}

% for dem plots
\usepackage{pgfplots}
%\pgfplotsset{compat=1.10}
\usepgfplotslibrary{fillbetween}

\usetikzlibrary{arrows}
% ---

\author{\authorinfotitle}
\title{\titleinfo}
\date{\today}

\pagestyle{fancy}
\fancyhf{}
\fancyhead[R]{\authorinfo}
\fancyhead[L]{Opti Hausaufgaben}
\fancyfoot[C]{\thepage}

\begin{document}
\maketitle
\begin{enumerate}
        % Aufgabe 1
    \item[\textbf{1.}]
        \begin{enumerate}
        \item[a)]
            Die folgende Tabelle stellt den Ablauf des Dijkstra Algorithmus dar. Die Zahl in den entsprechenden Feldern ist der bisher kürzeste Weg,
            der Buchstabe daneben der Vorgänger. Ein Wert ist unterstrichen, wenn er final ist.
            \begin{table}[h]
                \centering
                \begin{tabular}{c||c|c|c|c|c|c|c|c||l}
                    It. & $s$ & $a$ & $b$ & $c$ & $d$ & $e$ & $f$ & $g$ & $S$ \\
                    \hline
                    0 & $\underline{0}$ & $1s$ & $\infty$ & $\infty$ & $4s$ & $8s$ & $\infty$ & $\infty$ & $\{s\}$ \\
                    1 & $\underline{0}$ & $\underline{1s}$ & $3a$ & $\infty$ & $4s$ & $7a$ & $7a$ & $\infty$ & $\{s, a\}$ \\
                    2 & $\underline{0}$ & $\underline{1s}$ & $\underline{3a}$ & $4b$ & $4s$ & $7a$ & $5b$ & $\infty$ & $\{s, a, b\}$ \\
                    3 & $\underline{0}$ & $\underline{1s}$ & $\underline{3a}$ & $\underline{4b}$ & $4s$ & $7a$ & $5b$ & $8c$ & $\{s, a, b, c\}$ \\
                    4 & $\underline{0}$ & $\underline{1s}$ & $\underline{3a}$ & $\underline{4b}$ & $\underline{4s}$ & $7a$ & $5b$ & $8c$ & $\{s, a, b, c, d\}$ \\
                    5 & $\underline{0}$ & $\underline{1s}$ & $\underline{3a}$ & $\underline{4b}$ & $\underline{4s}$ & $6f$ & $\underline{5b}$ & $6f$ & $\{s, a, b, c, d, f\}$ \\
                    6 & $\underline{0}$ & $\underline{1s}$ & $\underline{3a}$ & $\underline{4b}$ & $\underline{4s}$ & $\underline{6f}$ & $\underline{5b}$ & $6f$ & $\{s, a, b, c, d, f, e\}$ \\
                    6 & $\underline{0}$ & $\underline{1s}$ & $\underline{3a}$ & $\underline{4b}$ & $\underline{4s}$ & $\underline{6f}$ & $\underline{5b}$ & $\underline{6f}$ & $\{s, a, b, c, d, f, e, g\}$ \\
                \end{tabular}
            \end{table}

            Den Kürzeste-Pfade-Baum kann man aus der obigen Tabelle ablesen:

            \begin{figure}[h]
                \centering
                \begin{tikzpicture}[
                        auto,
                        scale=2.5,
                    ]
                    \tikzstyle{vertex}=[
                        circle,
                        fill,
                    ]
                    \tikzstyle{edge}=[
                        ->,
                        >=stealth',
                    ]
                    \node [vertex, label=above:$s$] (s) at (0,1) {};
                    \node [vertex, label=above:$a$] (a) at (1,1) {};
                    \node [vertex, label=above:$b$] (b) at (2,1) {};
                    \node [vertex, label=above:$c$] (c) at (3,1) {};
                    \node [vertex, label=below:$d$] (d) at (0,0) {};
                    \node [vertex, label=below:$e$] (e) at (1,0) {};
                    \node [vertex, label=below:$f$] (f) at (2,0) {};
                    \node [vertex, label=below:$g$] (g) at (3,0) {};

                    \draw [edge] (s) to node {$1$} (a);
                    \draw [edge] (a) to node {$2$} (b);
                    \draw [edge] (b) to node {$1$} (c);
                    \draw [edge] (s) to node {$4$} (d);
                    \draw [edge] (b) to node {$2$} (f);
                    \draw [edge] (f) to node {$1$} (e);
                    \draw [edge] (f) to node {$1$} (g);
                \end{tikzpicture}
            \end{figure}
    \item[b)]
        Es genügt, ein einfaches Gegenbeispiel anzugeben:\\
        \begin{tabular}{c|c|c|c}
            $i$ & $v_i$ & $w_i$ & $q_i$ \\
            \hline
            $1$ & $5$ & $1$ & $5$ \\
            $2$ & $8$ & $2$ & $4$ \\
            $3$ & $14$ & $4$ & $3.5$
        \end{tabular}
        \\
        Der Rucksack habe eine Kapazität von $4$.\\
        Nach dem vorgeschlagenen Verfahren würden wir zuerst Gegenstand $1$ und dann Gegenstand $2$ einpacken, für ein
        Wert von $13$. Der Rucksack wäre dann zu $3/4$ gefüllt, aber kein anderer Gegenstand würde mehr hineinpassen. Wählten wir stattdessen einfach nur
        Gegenstand $3$, hätten wir einen Wert von $14$ mit einem vollständig gefüllten Rucksack.

\end{enumerate}

% Aufgabe 2
    \item[\textbf{2.}]
        \begin{enumerate}
        \item[a)]
            Eine mögliche Reihenfolge wäre:\\
            $(a,e)$, $(d,h)$, $(e,f)$, $(e,b)$, $(f,g)$, $(g,h)$, $(g,c)$.\\
            Es gibt unterschiedliche Möglichkeiten, da das Verhalten des Algorithmus bei Kanten gleichen Gewichts nicht genau
            definiert ist, sie können in willkürlicher Reihenfolge hinzugefügt werden. So können unterschiedliche Bäume entstehen.
            Der entstandene Baum sieht so aus:\\
            \begin{figure}[h]
                \centering
                \begin{tikzpicture}[
                        auto,
                        scale=2.5,
                    ]
                    \tikzstyle{vertex}=[
                        circle,
                        fill,
                    ]
                    \tikzstyle{edge}=[
                        -,
                    ]
                    \node [vertex, label=above:$a$] (a) at (0,1) {};
                    \node [vertex, label=above:$b$] (b) at (1,1) {};
                    \node [vertex, label=above:$c$] (c) at (2,1) {};
                    \node [vertex, label=above:$d$] (d) at (3,1) {};
                    \node [vertex, label=below:$e$] (e) at (0,0) {};
                    \node [vertex, label=below:$f$] (f) at (1,0) {};
                    \node [vertex, label=below:$g$] (g) at (2,0) {};
                    \node [vertex, label=below:$h$] (h) at (3,0) {};

                    \draw [edge] (a) to node {$1$} (e);
                    \draw [edge] (e) to node {$2$} (b);
                    \draw [edge] (e) to node {$1$} (f);
                    \draw [edge] (f) to node {$3$} (g);
                    \draw [edge] (g) to node {$4$} (c);
                    \draw [edge] (g) to node {$3$} (h);
                    \draw [edge] (h) to node {$1$} (d);
                \end{tikzpicture}
            \end{figure}

    \item[b)]
        (i): $(a,j)$, $(j,i)$, $(i,h)$, $(h,b)$, $(h,g)$, $(h,c)$, $(c,d)$, $(d,e)$, $(e,f)$\\
        (ii): $(e,d)$, $(g,h)$, $(h,i)$, $(h,b)$, $(d,c)$, $(c,h)$, $(a,j)$, $(i,j)$, $(e,f)$\\
        (iii): $(e,g)$, $(d,g)$, $(d,h)$, $(f,g)$, $(b,i)$, $(a,b)$, $(c,b)$, $(a,i)$\\
        Zufälligerweise ist der Graph für alle drei Verfahren hier der gleiche:\\

        \begin{figure}[h]
            \centering
            \begin{tikzpicture}[
                    auto,
                    scale=2.5,
                ]
                \tikzstyle{vertex}=[
                    circle,
                    fill,
                ]
                \tikzstyle{edge}=[
                    -,
                ]
                \node [vertex, label=above:$e$] (e) at (0,1) {};
                \node [vertex, label=above:$d$] (d) at (1,1) {};
                \node [vertex, label=above:$c$] (c) at (2,1) {};
                \node [vertex, label=above:$b$] (b) at (3,1) {};
                \node [vertex, label=below:$a$] (a) at (4,1) {};
                \node [vertex, label=below:$f$] (f) at (0,0) {};
                \node [vertex, label=below:$g$] (g) at (1,0) {};
                \node [vertex, label=below:$h$] (h) at (2,0) {};
                \node [vertex, label=below:$i$] (i) at (3,0) {};
                \node [vertex, label=below:$j$] (j) at (4,0) {};

                \draw [edge] (a) to node {$2$} (j);
                \draw [edge] (j) to node {$2$} (i);
                \draw [edge] (i) to node {$1$} (h);
                \draw [edge] (h) to node {$1$} (b);
                \draw [edge] (h) to node {$1$} (g);
                \draw [edge] (h) to node {$2$} (c);
                \draw [edge] (c) to node {$2$} (d);
                \draw [edge] (d) to node {$1$} (e);
                \draw [edge] (e) to node {$4$} (f);
            \end{tikzpicture}
        \end{figure}
\end{enumerate}

\end{enumerate}
\end{document}
