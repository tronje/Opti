\newcommand{\authorinfotitle}{Carolin Konietzny, 6523939, Gruppe 3\\ Tronje Krabbe, 6435002, Gruppe 7 \\ Julian Tobergte, 6414935, Gruppe 5}
\newcommand{\authorinfo}{Carolin Konietzny, Julian Tobergte, Tronje Krabbe}
\newcommand{\titleinfo}{Optimierung 05}
\newcommand{\qed}{\square}

\documentclass [a4paper,11pt]{article}
\usepackage[german,ngerman]{babel}
\usepackage[utf8]{inputenc}
\usepackage[T1]{fontenc}
\usepackage{lmodern}
\usepackage{amssymb}
\usepackage{mathtools}
\usepackage{amsmath}
\usepackage{enumerate}
%\usepackage{breqn}
\usepackage{fancyhdr}
\usepackage{multicol}
\usepackage{eurosym}

\usepackage[a4paper,left=2cm,width=13cm,right=3cm]{geometry}

% for dem plots
\usepackage{pgfplots}
%\pgfplotsset{compat=1.10}
\usepgfplotslibrary{fillbetween}
% ---

\author{\authorinfotitle}
\title{\titleinfo}
\date{\today}

\pagestyle{fancy}
\fancyhf{}
\fancyhead[R]{\authorinfo}
\fancyhead[L]{Opti Hausaufgaben}
\fancyfoot[C]{\thepage}

\begin{document}
\maketitle
    \begin{enumerate}
        % Aufgabe 1
        \item[\textbf{1.}]
        \begin{enumerate}
            \item[a)]
                \begin{align*}
                \begin{alignedat}{6}
                & \text{maximiere } & x_{47} &\ + &\ x_{57} &\ + &\ x_{67} & & & & \\
                & \rlap{unter den Nebenbedingungen} & & & & & & & & & \\
                && x_{04} &\ + &\ x_{34} &\ - &\ x_{47} &    &         &\ = &\ 0 \\
                && x_{35} &\ + &\ x_{25} &\ + &\ x_{65} &\ - &\ x_{57} &\ = &\ 0 \\
                && x_{26} &\ + &\ x_{16} &\ - &\ x_{65} &\ - &\ x_{67} &\ = &\ 0 \\
                && x_{03} &\ - &\ x_{34} &\ - &\ x_{35} &    &         &\ = &\ 0 \\
                && x_{02} &\ - &\ x_{21} &\ - &\ x_{25} &\ - &\ x_{26} &\ = &\ 0 \\
                && x_{01} &\ + &\ x_{21} &\ - &\ x_{16} &    &         &\ = &\ 0 \\
                &&        &    &         &    &         &    &\ x_{01} &\ \leq &\ 7 \\
                &&        &    &         &    &         &    &\ x_{02} &\ \leq &\ 1 \\
                &&        &    &         &    &         &    &\ x_{03} &\ \leq &\ 3 \\
                &&        &    &         &    &         &    &\ x_{04} &\ \leq &\ 2 \\
                &&        &    &         &    &         &    &\ x_{16} &\ \leq &\ 3 \\
                &&        &    &         &    &         &    &\ x_{21} &\ \leq &\ 4 \\
                &&        &    &         &    &         &    &\ x_{25} &\ \leq &\ 5 \\
                &&        &    &         &    &         &    &\ x_{26} &\ \leq &\ 2 \\
                &&        &    &         &    &         &    &\ x_{34} &\ \leq &\ 5 \\
                &&        &    &         &    &         &    &\ x_{35} &\ \leq &\ 4 \\
                &&        &    &         &    &         &    &\ x_{47} &\ \leq &\ 5 \\
                &&        &    &         &    &         &    &\ x_{57} &\ \leq &\ 2 \\
                &&        &    &         &    &         &    &\ x_{65} &\ \leq &\ 8 \\
                &&        &    &         &    &         &    &\ x_{67} &\ \leq &\ 3 \\
                && & & & & & & \llap{ $ x_{ij} $ } &\ \geq &\ 0
                \end{alignedat}
                \end{align*}

                \newpage

            \item[b)]
                \begin{align*}
                \begin{alignedat}{8}
                & \text{minimiere } \rlap{$ 5x_{01}+x_{06}+4x_{02}+3x_{03}+3x_{05}+3x_{16}+5x_{24} $} &&&&&&&&&&&\\
                & \rlap{$ +5x_{35}+6x_{12}+7x_{23}+4x_{46}+3x_{45}+2x_{56} $} &&&&&&&&&&&\\
                & \rlap{unter den Nebenbedingungen} & & & & & & & & & & & \\
                &&   x_{01} &\ + &\ x_{02} &\ + &\ x_{03} &\ + &\ x_{05} &\ +    &\ x_{06} &\ = &\ 0 \\
                &&&& x_{04} &\ + &\ x_{34} &\ - &\ x_{47} &    &         &\    = &\ 0 \\
                &&&& x_{35} &\ + &\ x_{25} &\ + &\ x_{65} &\ - &\ x_{57} &\    = &\ 0 \\
                &&&& x_{26} &\ + &\ x_{16} &\ - &\ x_{65} &\ - &\ x_{67} &\    = &\ 0 \\
                &&&& x_{03} &\ - &\ x_{34} &\ - &\ x_{35} &    &         &\    = &\ 0 \\
                &&&& x_{02} &\ - &\ x_{21} &\ - &\ x_{25} &\ - &\ x_{26} &\    = &\ 0 \\
                &&&& x_{01} &\ + &\ x_{21} &\ - &\ x_{16} &    &         &\    = &\ 0 \\
                &&&&        &    &         &    &         &    &\ x_{01} &\ \leq &\ 7 \\
                &&&&        &    &         &    &         &    &\ x_{02} &\ \leq &\ 1 \\
                &&&&        &    &         &    &         &    &\ x_{03} &\ \leq &\ 3 \\
                &&&&        &    &         &    &         &    &\ x_{04} &\ \leq &\ 2 \\
                &&&&        &    &         &    &         &    &\ x_{16} &\ \leq &\ 3 \\
                &&&&        &    &         &    &         &    &\ x_{21} &\ \leq &\ 4 \\
                &&&&        &    &         &    &         &    &\ x_{25} &\ \leq &\ 5 \\
                &&&&        &    &         &    &         &    &\ x_{26} &\ \leq &\ 2 \\
                &&&&        &    &         &    &         &    &\ x_{34} &\ \leq &\ 5 \\
                &&&&        &    &         &    &         &    &\ x_{35} &\ \leq &\ 4 \\
                &&&&        &    &         &    &         &    &\ x_{47} &\ \leq &\ 5 \\
                &&&&        &    &         &    &         &    &\ x_{57} &\ \leq &\ 2 \\
                &&&&        &    &         &    &         &    &\ x_{65} &\ \leq &\ 8 \\
                &&&&        &    &         &    &         &    &\ x_{67} &\ \leq &\ 3 \\
                &&&&        &    &         &    & & & \llap{ $ x_{ij} $ } &\ \geq &\ 0
                \end{alignedat}
                \end{align*}

            \item[c)]

        \end{enumerate}

        % Aufgabe 2
        \item[\textbf{2.}]
        \begin{enumerate}
            \item[a)]
                \begin{align*}
                145 \cdot 6 + 35 \cdot 14 + 20 \cdot 9 \\
                + 25 \cdot 6 + 35 \cdot 15 + 55 \cdot 11 \\
                + 80 \cdot 10 + 140 \cdot 16 + 20 \cdot 13 \\
                =6120
                \end{align*}
                Der Plan führt also zu einem Gewinn von 6120 \euro.
                \newpage
            \item[b)]
                Es sei der erste Index (A, B oder C) entsprechend der Eissorte, und der zweite Index der Veredelung (1 ist normal, 2 ist veredelt, und 3 ist mit
                Überstunden veredelt).\\
                $x_{C1}$ beispielsweise ist dann Eissorte C, unveredelt.
                % \begin{align*}
                % \begin{alignedat}{12}
                % & \text{maximiere } & 6x_{A1} &\ + &\ 14x_{A2} &\ + &\ 9x_{A3} &\ + &\ 6x_{B1} &\ + &\ 15x_{B2} &\ + &\ 11x_{B3} &\ + &\ 10x_{C1} &\ + &\ 16x_{C2} &\ + &\ 13x_{C3} \\
                % & \rlap{unter den Nebenbedingungen} & & & & & & & & & & & & & & & & & \\
                % &&  x_{A1} &\ + &\ x_{A2} &\ + &\ x_{A3} &    &      & & & & & & & & &\ \leq &\  200 \\
                % && -x_{A1} &\ - &\ x_{A2} &\ - &\ x_{A3} &    &      & & & & & & & & &\ \leq &\ -200 \\
                % &&  &  &  & & &\ x_{B1} &\ + &\ x_{B2} &\ + &\ x_{B3} & & & & & & & &\ \leq &\ 115 \\
                % &&  &  &  & &\ - &\ x_{B1} &\ - &\ x_{B2} &\ - &\ x_{B3} & & & & & & & &\ \leq &\ -115 \\
                % &&  &  &  & &  & & & & & & & &\ x_{C1} &\ + &\ x_{C2} &\ + &\ x_{C3} &\ \leq &\ 240 \\
                % &&  &  &  & &  & & & & & & &\ - &\ x_{C1} &\ - &\ x_{C2} &\ - &\ x_{C3} &\ \leq &\ -240 \\
                % && & & & & & & & & & & & & & & \llap{$x_{A1}, x_{A2}, x_{A3},x_{B1}, x_{B2}, x_{B3}, x_{C1}, x_{C2}, x_{C3}$} &\ \geq &\ 0
                % \end{alignedat}
                % \end{align*}

                \begin{align*}
                \begin{alignedat}{12}
                & \text{maximiere } \rlap{ $ 6x_{A1} + 14x_{A2} + 9x_{A3} + 6x_{B1} + 15x_{B2} + 11x_{B3} + 10x_{C1} + 16x_{C2} + 13x_{C3} $} &&&&&&&&&&&&&&&&&&& \\
                & \rlap{unter den Nebenbedingungen} & & & & & & & & & & & & & & & & & & & \\
                &&  x_{A1} &\ + &\ x_{A2} &\ + &\ x_{A3} &    &  & & & & & & & & & & &\ \leq &\  200 \\
                && -x_{A1} &\ - &\ x_{A2} &\ - &\ x_{A3} &    &  & & & & & & & & & & &\ \leq &\ -200 \\
                &&  &  &  & & &\ + &\ x_{B1} &\ + &\ x_{B2} &\ + &\ x_{B3} & & & & & & &\ \leq &\ 115 \\
                &&  &  &  & & &\ - &\ x_{B1} &\ - &\ x_{B2} &\ - &\ x_{B3} & & & & & & &\ \leq &\ -115 \\
                &&  &  &  & &  & & & & & & & &\ x_{C1} &\ + &\ x_{C2} &\ + &\ x_{C3} &\ \leq &\ 240 \\
                &&  &  &  & &  & & & & & & &\ - &\ x_{C1} &\ - &\ x_{C2} &\ - &\ x_{C3} &\ \leq &\ -240 \\
                &&  &  &\ x_{A2} & & & & &\ + &\ x_{B2} & & & & &\ + &\ x_{C2} & & &\ \leq &\  210 \\
                &&  &  &  & &\ x_{A3} & & & & &\ + &\ x_{B3} & & & & &\ + &\ x_{C3} &\ \leq &\ 125 \\
                && & & & & & & & & & & & & & & & & \llap{$x_{A1}, x_{A2}, x_{A3},x_{B1}, x_{B2}, x_{B3}, x_{C1}, x_{C2}, x_{C3}$} &\ \geq &\ 0
                \end{alignedat}
                \end{align*}

        \end{enumerate}

    \end{enumerate}


\end{document}
